% texinfo.tex -- TeX macros to handle Texinfo files.
% 
% Load plain if necessary, i.e., if running under initex.
\expandafter\ifx\csname fmtname\endcsname\relax\input plain\fi
%
\def\texinfoversion{2013-02-01.11}
%
% Copyright 1985, 1986, 1988, 1990, 1991, 1992, 1993, 1994, 1995,
% 1996, 1997, 1998, 1999, 2000, 2001, 2002, 2003, 2004, 2005, 2006,
% 2007, 2008, 2009, 2010, 2011, 2012, 2013 Free Software Foundation, Inc.
%
% This texinfo.tex file is free software: you can redistribute it and/or
% modify it under the terms of the GNU General Public License as
% published by the Free Software Foundation, either version 3 of the
% License, or (at your option) any later version.
%
% This texinfo.tex file is distributed in the hope that it will be
% useful, but WITHOUT ANY WARRANTY; without even the implied warranty
% of MERCHANTABILITY or FITNESS FOR A PARTICULAR PURPOSE.  See the GNU
% General Public License for more details.
%
% You should have received a copy of the GNU General Public License
% along with this program.  If not, see <http://www.gnu.org/licenses/>.
%
% As a special exception, when this file is read by TeX when processing
% a Texinfo source document, you may use the result without
% restriction. This Exception is an additional permission under section 7
% of the GNU General Public License, version 3 ("GPLv3").
%
% Please try the latest version of texinfo.tex before submitting bug
% reports; you can get the latest version from:
%   http://ftp.gnu.org/gnu/texinfo/ (the Texinfo release area), or
%   http://ftpmirror.gnu.org/texinfo/ (same, via a mirror), or
%   http://www.gnu.org/software/texinfo/ (the Texinfo home page)
% The texinfo.tex in any given distribution could well be out
% of date, so if that's what you're using, please check.
%
% Send bug reports to bug-texinfo@gnu.org.  Please include including a
% complete document in each bug report with which we can reproduce the
% problem.  Patches are, of course, greatly appreciated.
%
% To process a Texinfo manual with TeX, it's most reliable to use the
% texi2dvi shell script that comes with the distribution.  For a simple
% manual foo.texi, however, you can get away with this:
%   tex foo.texi
%   texindex foo.??
%   tex foo.texi
%   tex foo.texi
%   dvips foo.dvi -o  # or whatever; this makes foo.ps.
% The extra TeX runs get the cross-reference information correct.
% Sometimes one run after texindex suffices, and sometimes you need more
% than two; texi2dvi does it as many times as necessary.
%
% It is possible to adapt texinfo.tex for other languages, to some
% extent.  You can get the existing language-specific files from the
% full Texinfo distribution.
%
% The GNU Texinfo home page is http://www.gnu.org/software/texinfo.


\message{Loading texinfo [version \texinfoversion]:}

% If in a .fmt file, print the version number
% and turn on active characters that we couldn't do earlier because
% they might have appeared in the input file name.
\everyjob{\message{[Texinfo version \texinfoversion]}%
  \catcode`+=\active \catcode`\_=\active}

\chardef\other=12

% We never want plain's \outer definition of \+ in Texinfo.
% For @tex, we can use \tabalign.
\let\+ = \relax

% Save some plain tex macros whose names we will redefine.
\let\ptexb=\b
\let\ptexbullet=\bullet
\let\ptexc=\c
\let\ptexcomma=\,
\let\ptexdot=\.
\let\ptexdots=\dots
\let\ptexend=\end
\let\ptexequiv=\equiv
\let\ptexexclam=\!
\let\ptexfootnote=\footnote
\let\ptexgtr=>
\let\ptexhat=^
\let\ptexi=\i
\let\ptexindent=\indent
\let\ptexinsert=\insert
\let\ptexlbrace=\{
\let\ptexless=<
\let\ptexnewwrite\newwrite
\let\ptexnoindent=\noindent
\let\ptexplus=+
\let\ptexraggedright=\raggedright
\let\ptexrbrace=\}
\let\ptexslash=\/
\let\ptexstar=\*
\let\ptext=\t
\let\ptextop=\top
{\catcode`\'=\active \global\let\ptexquoteright'}% active in plain's math mode

% If this character appears in an error message or help string, it
% starts a new line in the output.
\newlinechar = `^^J

% Use TeX 3.0's \inputlineno to get the line number, for better error
% messages, but if we're using an old version of TeX, don't do anything.
%
\ifx\inputlineno\thisisundefined
  \let\linenumber = \empty % Pre-3.0.
\else
  \def\linenumber{l.\the\inputlineno:\space}
\fi

% Set up fixed words for English if not already set.
\ifx\putwordAppendix\undefined  \gdef\putwordAppendix{Appendix}\fi
\ifx\putwordChapter\undefined   \gdef\putwordChapter{Chapter}\fi
\ifx\putworderror\undefined     \gdef\putworderror{error}\fi
\ifx\putwordfile\undefined      \gdef\putwordfile{file}\fi
\ifx\putwordin\undefined        \gdef\putwordin{in}\fi
\ifx\putwordIndexIsEmpty\undefined       \gdef\putwordIndexIsEmpty{(Index is empty)}\fi
\ifx\putwordIndexNonexistent\undefined   \gdef\putwordIndexNonexistent{(Index is nonexistent)}\fi
\ifx\putwordInfo\undefined      \gdef\putwordInfo{Info}\fi
\ifx\putwordInstanceVariableof\undefined \gdef\putwordInstanceVariableof{Instance Variable of}\fi
\ifx\putwordMethodon\undefined  \gdef\putwordMethodon{Method on}\fi
\ifx\putwordNoTitle\undefined   \gdef\putwordNoTitle{No Title}\fi
\ifx\putwordof\undefined        \gdef\putwordof{of}\fi
\ifx\putwordon\undefined        \gdef\putwordon{on}\fi
\ifx\putwordpage\undefined      \gdef\putwordpage{page}\fi
\ifx\putwordsection\undefined   \gdef\putwordsection{section}\fi
\ifx\putwordSection\undefined   \gdef\putwordSection{Section}\fi
\ifx\putwordsee\undefined       \gdef\putwordsee{see}\fi
\ifx\putwordSee\undefined       \gdef\putwordSee{See}\fi
\ifx\putwordShortTOC\undefined  \gdef\putwordShortTOC{Short Contents}\fi
\ifx\putwordTOC\undefined       \gdef\putwordTOC{Table of Contents}\fi
%
\ifx\putwordMJan\undefined \gdef\putwordMJan{January}\fi
\ifx\putwordMFeb\undefined \gdef\putwordMFeb{February}\fi
\ifx\putwordMMar\undefined \gdef\putwordMMar{March}\fi
\ifx\putwordMApr\undefined \gdef\putwordMApr{April}\fi
\ifx\putwordMMay\undefined \gdef\putwordMMay{May}\fi
\ifx\putwordMJun\undefined \gdef\putwordMJun{June}\fi
\ifx\putwordMJul\undefined \gdef\putwordMJul{July}\fi
\ifx\putwordMAug\undefined \gdef\putwordMAug{August}\fi
\ifx\putwordMSep\undefined \gdef\putwordMSep{September}\fi
\ifx\putwordMOct\undefined \gdef\putwordMOct{October}\fi
\ifx\putwordMNov\undefined \gdef\putwordMNov{November}\fi
\ifx\putwordMDec\undefined \gdef\putwordMDec{December}\fi
%
\ifx\putwordDefmac\undefined    \gdef\putwordDefmac{Macro}\fi
\ifx\putwordDefspec\undefined   \gdef\putwordDefspec{Special Form}\fi
\ifx\putwordDefvar\undefined    \gdef\putwordDefvar{Variable}\fi
\ifx\putwordDefopt\undefined    \gdef\putwordDefopt{User Option}\fi
\ifx\putwordDeffunc\undefined   \gdef\putwordDeffunc{Function}\fi

% Since the category of space is not known, we have to be careful.
\chardef\spacecat = 10
\def\spaceisspace{\catcode`\ =\spacecat}

% sometimes characters are active, so we need control sequences.
\chardef\ampChar   = `\&
\chardef\colonChar = `\:
\chardef\commaChar = `\,
\chardef\dashChar  = `\-
\chardef\dotChar   = `\.
\chardef\exclamChar= `\!
\chardef\hashChar  = `\#
\chardef\lquoteChar= `\`
\chardef\questChar = `\?
\chardef\rquoteChar= `\'
\chardef\semiChar  = `\;
\chardef\slashChar = `\/
\chardef\underChar = `\_

% Ignore a token.
%
\def\gobble#1{}

% The following is used inside several \edef's.
\def\makecsname#1{\expandafter\noexpand\csname#1\endcsname}

% Hyphenation fixes.
\hyphenation{
  Flor-i-da Ghost-script Ghost-view Mac-OS Post-Script
  ap-pen-dix bit-map bit-maps
  data-base data-bases eshell fall-ing half-way long-est man-u-script
  man-u-scripts mini-buf-fer mini-buf-fers over-view par-a-digm
  par-a-digms rath-er rec-tan-gu-lar ro-bot-ics se-vere-ly set-up spa-ces
  spell-ing spell-ings
  stand-alone strong-est time-stamp time-stamps which-ever white-space
  wide-spread wrap-around
}

% Margin to add to right of even pages, to left of odd pages.
\newdimen\bindingoffset
\newdimen\normaloffset
\newdimen\pagewidth \newdimen\pageheight

% For a final copy, take out the rectangles
% that mark overfull boxes (in case you have decided
% that the text looks ok even though it passes the margin).
%
\def\finalout{\overfullrule=0pt }

% Sometimes it is convenient to have everything in the transcript file
% and nothing on the terminal.  We don't just call \tracingall here,
% since that produces some useless output on the terminal.  We also make
% some effort to order the tracing commands to reduce output in the log
% file; cf. trace.sty in LaTeX.
%
\def\gloggingall{\begingroup \globaldefs = 1 \loggingall \endgroup}%
\def\loggingall{%
  \tracingstats2
  \tracingpages1
  \tracinglostchars2  % 2 gives us more in etex
  \tracingparagraphs1
  \tracingoutput1
  \tracingmacros2
  \tracingrestores1
  \showboxbreadth\maxdimen \showboxdepth\maxdimen
  \ifx\eTeXversion\thisisundefined\else % etex gives us more logging
    \tracingscantokens1
    \tracingifs1
    \tracinggroups1
    \tracingnesting2
    \tracingassigns1
  \fi
  \tracingcommands3  % 3 gives us more in etex
  \errorcontextlines16
}%

% @errormsg{MSG}.  Do the index-like expansions on MSG, but if things
% aren't perfect, it's not the end of the world, being an error message,
% after all.
% 
\def\errormsg{\begingroup \indexnofonts \doerrormsg}
\def\doerrormsg#1{\errmessage{#1}}

% add check for \lastpenalty to plain's definitions.  If the last thing
% we did was a \nobreak, we don't want to insert more space.
%
\def\smallbreak{\ifnum\lastpenalty<10000\par\ifdim\lastskip<\smallskipamount
  \removelastskip\penalty-50\smallskip\fi\fi}
\def\medbreak{\ifnum\lastpenalty<10000\par\ifdim\lastskip<\medskipamount
  \removelastskip\penalty-100\medskip\fi\fi}
\def\bigbreak{\ifnum\lastpenalty<10000\par\ifdim\lastskip<\bigskipamount
  \removelastskip\penalty-200\bigskip\fi\fi}

% Do @cropmarks to get crop marks.
%
\newif\ifcropmarks
\let\cropmarks = \cropmarkstrue
%
% Dimensions to add cropmarks at corners.
% Added by P. A. MacKay, 12 Nov. 1986
%
\newdimen\outerhsize \newdimen\outervsize % set by the paper size routines
\newdimen\cornerlong  \cornerlong=1pc
\newdimen\cornerthick \cornerthick=.3pt
\newdimen\topandbottommargin \topandbottommargin=.75in

% Output a mark which sets \thischapter, \thissection and \thiscolor.
% We dump everything together because we only have one kind of mark.
% This works because we only use \botmark / \topmark, not \firstmark.
%
% A mark contains a subexpression of the \ifcase ... \fi construct.
% \get*marks macros below extract the needed part using \ifcase.
%
% Another complication is to let the user choose whether \thischapter
% (\thissection) refers to the chapter (section) in effect at the top
% of a page, or that at the bottom of a page.  The solution is
% described on page 260 of The TeXbook.  It involves outputting two
% marks for the sectioning macros, one before the section break, and
% one after.  I won't pretend I can describe this better than DEK...
\def\domark{%
  \toks0=\expandafter{\lastchapterdefs}%
  \toks2=\expandafter{\lastsectiondefs}%
  \toks4=\expandafter{\prevchapterdefs}%
  \toks6=\expandafter{\prevsectiondefs}%
  \toks8=\expandafter{\lastcolordefs}%
  \mark{%
                   \the\toks0 \the\toks2
      \noexpand\or \the\toks4 \the\toks6
    \noexpand\else \the\toks8
  }%
}
% \topmark doesn't work for the very first chapter (after the title
% page or the contents), so we use \firstmark there -- this gets us
% the mark with the chapter defs, unless the user sneaks in, e.g.,
% @setcolor (or @url, or @link, etc.) between @contents and the very
% first @chapter.
\def\gettopheadingmarks{%
  \ifcase0\topmark\fi
  \ifx\thischapter\empty \ifcase0\firstmark\fi \fi
}
\def\getbottomheadingmarks{\ifcase1\botmark\fi}
\def\getcolormarks{\ifcase2\topmark\fi}

% Avoid "undefined control sequence" errors.
\def\lastchapterdefs{}
\def\lastsectiondefs{}
\def\prevchapterdefs{}
\def\prevsectiondefs{}
\def\lastcolordefs{}

% Main output routine.
\chardef\PAGE = 255
\output = {\onepageout{\pagecontents\PAGE}}

\newbox\headlinebox
\newbox\footlinebox

% \onepageout takes a vbox as an argument.  Note that \pagecontents
% does insertions, but you have to call it yourself.
\def\onepageout#1{%
  \ifcropmarks \hoffset=0pt \else \hoffset=\normaloffset \fi
  %
  \ifodd\pageno  \advance\hoffset by \bindingoffset
  \else \advance\hoffset by -\bindingoffset\fi
  %
  % Do this outside of the \shipout so @code etc. will be expanded in
  % the headline as they should be, not taken literally (outputting ''code).
  \ifodd\pageno \getoddheadingmarks \else \getevenheadingmarks \fi
  \setbox\headlinebox = \vbox{\let\hsize=\pagewidth \makeheadline}%
  \ifodd\pageno \getoddfootingmarks \else \getevenfootingmarks \fi
  \setbox\footlinebox = \vbox{\let\hsize=\pagewidth \makefootline}%
  %
  {%
    % Have to do this stuff outside the \shipout because we want it to
    % take effect in \write's, yet the group defined by the \vbox ends
    % before the \shipout runs.
    %
    \indexdummies         % don't expand commands in the output.
    \normalturnoffactive  % \ in index entries must not stay \, e.g., if
               % the page break happens to be in the middle of an example.
               % We don't want .vr (or whatever) entries like this:
               % \entry{{\tt \indexbackslash }acronym}{32}{\code {\acronym}}
               % "\acronym" won't work when it's read back in;
               % it needs to be
               % {\code {{\tt \backslashcurfont }acronym}
    \shipout\vbox{%
      % Do this early so pdf references go to the beginning of the page.
      \ifpdfmakepagedest \pdfdest name{\the\pageno} xyz\fi
      %
      \ifcropmarks \vbox to \outervsize\bgroup
        \hsize = \outerhsize
        \vskip-\topandbottommargin
        \vtop to0pt{%
          \line{\ewtop\hfil\ewtop}%
          \nointerlineskip
          \line{%
            \vbox{\moveleft\cornerthick\nstop}%
            \hfill
            \vbox{\moveright\cornerthick\nstop}%
          }%
          \vss}%
        \vskip\topandbottommargin
        \line\bgroup
          \hfil % center the page within the outer (page) hsize.
          \ifodd\pageno\hskip\bindingoffset\fi
          \vbox\bgroup
      \fi
      %
      \unvbox\headlinebox
      \pagebody{#1}%
      \ifdim\ht\footlinebox > 0pt
        % Only leave this space if the footline is nonempty.
        % (We lessened \vsize for it in \oddfootingyyy.)
        % The \baselineskip=24pt in plain's \makefootline has no effect.
        \vskip 24pt
        \unvbox\footlinebox
      \fi
      %
      \ifcropmarks
          \egroup % end of \vbox\bgroup
        \hfil\egroup % end of (centering) \line\bgroup
        \vskip\topandbottommargin plus1fill minus1fill
        \boxmaxdepth = \cornerthick
        \vbox to0pt{\vss
          \line{%
            \vbox{\moveleft\cornerthick\nsbot}%
            \hfill
            \vbox{\moveright\cornerthick\nsbot}%
          }%
          \nointerlineskip
          \line{\ewbot\hfil\ewbot}%
        }%
      \egroup % \vbox from first cropmarks clause
      \fi
    }% end of \shipout\vbox
  }% end of group with \indexdummies
  \advancepageno
  \ifnum\outputpenalty>-20000 \else\dosupereject\fi
}

\newinsert\margin \dimen\margin=\maxdimen

\def\pagebody#1{\vbox to\pageheight{\boxmaxdepth=\maxdepth #1}}
{\catcode`\@ =11
\gdef\pagecontents#1{\ifvoid\topins\else\unvbox\topins\fi
% marginal hacks, juha@viisa.uucp (Juha Takala)
\ifvoid\margin\else % marginal info is present
  \rlap{\kern\hsize\vbox to\z@{\kern1pt\box\margin \vss}}\fi
\dimen@=\dp#1\relax \unvbox#1\relax
\ifvoid\footins\else\vskip\skip\footins\footnoterule \unvbox\footins\fi
\ifr@ggedbottom \kern-\dimen@ \vfil \fi}
}

% Here are the rules for the cropmarks.  Note that they are
% offset so that the space between them is truly \outerhsize or \outervsize
% (P. A. MacKay, 12 November, 1986)
%
\def\ewtop{\vrule height\cornerthick depth0pt width\cornerlong}
\def\nstop{\vbox
  {\hrule height\cornerthick depth\cornerlong width\cornerthick}}
\def\ewbot{\vrule height0pt depth\cornerthick width\cornerlong}
\def\nsbot{\vbox
  {\hrule height\cornerlong depth\cornerthick width\cornerthick}}

% Parse an argument, then pass it to #1.  The argument is the rest of
% the input line (except we remove a trailing comment).  #1 should be a
% macro which expects an ordinary undelimited TeX argument.
%
\def\parsearg{\parseargusing{}}
\def\parseargusing#1#2{%
  \def\argtorun{#2}%
  \begingroup
    \obeylines
    \spaceisspace
    #1%
    \parseargline\empty% Insert the \empty token, see \finishparsearg below.
}

{\obeylines %
  \gdef\parseargline#1^^M{%
    \endgroup % End of the group started in \parsearg.
    \argremovecomment #1\comment\ArgTerm%
  }%
}

% First remove any @comment, then any @c comment.
\def\argremovecomment#1\comment#2\ArgTerm{\argremovec #1\c\ArgTerm}
\def\argremovec#1\c#2\ArgTerm{\argcheckspaces#1\^^M\ArgTerm}

% Each occurrence of `\^^M' or `<space>\^^M' is replaced by a single space.
%
% \argremovec might leave us with trailing space, e.g.,
%    @end itemize  @c foo
% This space token undergoes the same procedure and is eventually removed
% by \finishparsearg.
%
\def\argcheckspaces#1\^^M{\argcheckspacesX#1\^^M \^^M}
\def\argcheckspacesX#1 \^^M{\argcheckspacesY#1\^^M}
\def\argcheckspacesY#1\^^M#2\^^M#3\ArgTerm{%
  \def\temp{#3}%
  \ifx\temp\empty
    % Do not use \next, perhaps the caller of \parsearg uses it; reuse \temp:
    \let\temp\finishparsearg
  \else
    \let\temp\argcheckspaces
  \fi
  % Put the space token in:
  \temp#1 #3\ArgTerm
}

% If a _delimited_ argument is enclosed in braces, they get stripped; so
% to get _exactly_ the rest of the line, we had to prevent such situation.
% We prepended an \empty token at the very beginning and we expand it now,
% just before passing the control to \argtorun.
% (Similarly, we have to think about #3 of \argcheckspacesY above: it is
% either the null string, or it ends with \^^M---thus there is no danger
% that a pair of braces would be stripped.
%
% But first, we have to remove the trailing space token.
%
\def\finishparsearg#1 \ArgTerm{\expandafter\argtorun\expandafter{#1}}

% \parseargdef\foo{...}
%	is roughly equivalent to
% \def\foo{\parsearg\Xfoo}
% \def\Xfoo#1{...}
%
% Actually, I use \csname\string\foo\endcsname, ie. \\foo, as it is my
% favourite TeX trick.  --kasal, 16nov03

\def\parseargdef#1{%
  \expandafter \doparseargdef \csname\string#1\endcsname #1%
}
\def\doparseargdef#1#2{%
  \def#2{\parsearg#1}%
  \def#1##1%
}

% Several utility definitions with active space:
{
  \obeyspaces
  \gdef\obeyedspace{ }

  % Make each space character in the input produce a normal interword
  % space in the output.  Don't allow a line break at this space, as this
  % is used only in environments like @example, where each line of input
  % should produce a line of output anyway.
  %
  \gdef\sepspaces{\obeyspaces\let =\tie}

  % If an index command is used in an @example environment, any spaces
  % therein should become regular spaces in the raw index file, not the
  % expansion of \tie (\leavevmode \penalty \@M \ ).
  \gdef\unsepspaces{\let =\space}
}


\def\flushcr{\ifx\par\lisppar \def\next##1{}\else \let\next=\relax \fi \next}

% Define the framework for environments in texinfo.tex.  It's used like this:
%
%   \envdef\foo{...}
%   \def\Efoo{...}
%
% It's the responsibility of \envdef to insert \begingroup before the
% actual body; @end closes the group after calling \Efoo.  \envdef also
% defines \thisenv, so the current environment is known; @end checks
% whether the environment name matches.  The \checkenv macro can also be
% used to check whether the current environment is the one expected.
%
% Non-false conditionals (@iftex, @ifset) don't fit into this, so they
% are not treated as environments; they don't open a group.  (The
% implementation of @end takes care not to call \endgroup in this
% special case.)


% At run-time, environments start with this:
\def\startenvironment#1{\begingroup\def\thisenv{#1}}
% initialize
\let\thisenv\empty

% ... but they get defined via ``\envdef\foo{...}'':
\long\def\envdef#1#2{\def#1{\startenvironment#1#2}}
\def\envparseargdef#1#2{\parseargdef#1{\startenvironment#1#2}}

% Check whether we're in the right environment:
\def\checkenv#1{%
  \def\temp{#1}%
  \ifx\thisenv\temp
  \else
    \badenverr
  \fi
}

% Environment mismatch, #1 expected:
\def\badenverr{%
  \errhelp = \EMsimple
  \errmessage{This command can appear only \inenvironment\temp,
    not \inenvironment\thisenv}%
}
\def\inenvironment#1{%
  \ifx#1\empty
    outside of any environment%
  \else
    in environment \expandafter\string#1%
  \fi
}

% @end foo executes the definition of \Efoo.
% But first, it executes a specialized version of \checkenv
%
\parseargdef\end{%
  \if 1\csname iscond.#1\endcsname
  \else
    % The general wording of \badenverr may not be ideal.
    \expandafter\checkenv\csname#1\endcsname
    \csname E#1\endcsname
    \endgroup
  \fi
}

\newhelp\EMsimple{Press RETURN to continue.}


% Be sure we're in horizontal mode when doing a tie, since we make space
% equivalent to this in @example-like environments. Otherwise, a space
% at the beginning of a line will start with \penalty -- and
% since \penalty is valid in vertical mode, we'd end up putting the
% penalty on the vertical list instead of in the new paragraph.
{\catcode`@ = 11
 % Avoid using \@M directly, because that causes trouble
 % if the definition is written into an index file.
 \global\let\tiepenalty = \@M
 \gdef\tie{\leavevmode\penalty\tiepenalty\ }
}

% @: forces normal size whitespace following.
\def\:{\spacefactor=1000 }

% @* forces a line break.
\def\*{\unskip\hfil\break\hbox{}\ignorespaces}

% @/ allows a line break.
\let\/=\allowbreak

% @. is an end-of-sentence period.
\def\.{.\spacefactor=\endofsentencespacefactor\space}

% @! is an end-of-sentence bang.
\def\!{!\spacefactor=\endofsentencespacefactor\space}

% @? is an end-of-sentence query.
\def\?{?\spacefactor=\endofsentencespacefactor\space}

% @frenchspacing on|off  says whether to put extra space after punctuation.
%
\def\onword{on}
\def\offword{off}
%
\parseargdef\frenchspacing{%
  \def\temp{#1}%
  \ifx\temp\onword \plainfrenchspacing
  \else\ifx\temp\offword \plainnonfrenchspacing
  \else
    \errhelp = \EMsimple
    \errmessage{Unknown @frenchspacing option `\temp', must be on|off}%
  \fi\fi
}

% @w prevents a word break.  Without the \leavevmode, @w at the
% beginning of a paragraph, when TeX is still in vertical mode, would
% produce a whole line of output instead of starting the paragraph.
\def\w#1{\leavevmode\hbox{#1}}

% @group ... @end group forces ... to be all on one page, by enclosing
% it in a TeX vbox.  We use \vtop instead of \vbox to construct the box
% to keep its height that of a normal line.  According to the rules for
% \topskip (p.114 of the TeXbook), the glue inserted is
% max (\topskip - \ht (first item), 0).  If that height is large,
% therefore, no glue is inserted, and the space between the headline and
% the text is small, which looks bad.
%
% Another complication is that the group might be very large.  This can
% cause the glue on the previous page to be unduly stretched, because it
% does not have much material.  In this case, it's better to add an
% explicit \vfill so that the extra space is at the bottom.  The
% threshold for doing this is if the group is more than \vfilllimit
% percent of a page (\vfilllimit can be changed inside of @tex).
%
\newbox\groupbox
\def\vfilllimit{0.7}
%
\envdef\group{%
  \ifnum\catcode`\^^M=\active \else
    \errhelp = \groupinvalidhelp
    \errmessage{@group invalid in context where filling is enabled}%
  \fi
  \startsavinginserts
  %
  \setbox\groupbox = \vtop\bgroup
    % Do @comment since we are called inside an environment such as
    % @example, where each end-of-line in the input causes an
    % end-of-line in the output.  We don't want the end-of-line after
    % the `@group' to put extra space in the output.  Since @group
    % should appear on a line by itself (according to the Texinfo
    % manual), we don't worry about eating any user text.
    \comment
}
%
% The \vtop produces a box with normal height and large depth; thus, TeX puts
% \baselineskip glue before it, and (when the next line of text is done)
% \lineskip glue after it.  Thus, space below is not quite equal to space
% above.  But it's pretty close.
\def\Egroup{%
    % To get correct interline space between the last line of the group
    % and the first line afterwards, we have to propagate \prevdepth.
    \endgraf % Not \par, as it may have been set to \lisppar.
    \global\dimen1 = \prevdepth
  \egroup           % End the \vtop.
  % \dimen0 is the vertical size of the group's box.
  \dimen0 = \ht\groupbox  \advance\dimen0 by \dp\groupbox
  % \dimen2 is how much space is left on the page (more or less).
  \dimen2 = \pageheight   \advance\dimen2 by -\pagetotal
  % if the group doesn't fit on the current page, and it's a big big
  % group, force a page break.
  \ifdim \dimen0 > \dimen2
    \ifdim \pagetotal < \vfilllimit\pageheight
      \page
    \fi
  \fi
  \box\groupbox
  \prevdepth = \dimen1
  \checkinserts
}
%
% TeX puts in an \escapechar (i.e., `@') at the beginning of the help
% message, so this ends up printing `@group can only ...'.
%
\newhelp\groupinvalidhelp{%
group can only be used in environments such as @example,^^J%
where each line of input produces a line of output.}

% @need space-in-mils
% forces a page break if there is not space-in-mils remaining.

\newdimen\mil  \mil=0.001in

\parseargdef\need{%
  % Ensure vertical mode, so we don't make a big box in the middle of a
  % paragraph.
  \par
  %
  % If the @need value is less than one line space, it's useless.
  \dimen0 = #1\mil
  \dimen2 = \ht\strutbox
  \advance\dimen2 by \dp\strutbox
  \ifdim\dimen0 > \dimen2
    %
    % Do a \strut just to make the height of this box be normal, so the
    % normal leading is inserted relative to the preceding line.
    % And a page break here is fine.
    \vtop to #1\mil{\strut\vfil}%
    %
    % TeX does not even consider page breaks if a penalty added to the
    % main vertical list is 10000 or more.  But in order to see if the
    % empty box we just added fits on the page, we must make it consider
    % page breaks.  On the other hand, we don't want to actually break the
    % page after the empty box.  So we use a penalty of 9999.
    %
    % There is an extremely small chance that TeX will actually break the
    % page at this \penalty, if there are no other feasible breakpoints in
    % sight.  (If the user is using lots of big @group commands, which
    % almost-but-not-quite fill up a page, TeX will have a hard time doing
    % good page breaking, for example.)  However, I could not construct an
    % example where a page broke at this \penalty; if it happens in a real
    % document, then we can reconsider our strategy.
    \penalty9999
    %
    % Back up by the size of the box, whether we did a page break or not.
    \kern -#1\mil
    %
    % Do not allow a page break right after this kern.
    \nobreak
  \fi
}

% @br   forces paragraph break (and is undocumented).

\let\br = \par

% @page forces the start of a new page.
%
\def\page{\par\vfill\supereject}

% @exdent text....
% outputs text on separate line in roman font, starting at standard page margin

% This records the amount of indent in the innermost environment.
% That's how much \exdent should take out.
\newskip\exdentamount

% This defn is used inside fill environments such as @defun.
\parseargdef\exdent{\hfil\break\hbox{\kern -\exdentamount{\rm#1}}\hfil\break}

% This defn is used inside nofill environments such as @example.
\parseargdef\nofillexdent{{\advance \leftskip by -\exdentamount
  \leftline{\hskip\leftskip{\rm#1}}}}

% @inmargin{WHICH}{TEXT} puts TEXT in the WHICH margin next to the current
% paragraph.  For more general purposes, use the \margin insertion
% class.  WHICH is `l' or `r'.  Not documented, written for gawk manual.
%
\newskip\inmarginspacing \inmarginspacing=1cm
\def\strutdepth{\dp\strutbox}
%
\def\doinmargin#1#2{\strut\vadjust{%
  \nobreak
  \kern-\strutdepth
  \vtop to \strutdepth{%
    \baselineskip=\strutdepth
    \vss
    % if you have multiple lines of stuff to put here, you'll need to
    % make the vbox yourself of the appropriate size.
    \ifx#1l%
      \llap{\ignorespaces #2\hskip\inmarginspacing}%
    \else
      \rlap{\hskip\hsize \hskip\inmarginspacing \ignorespaces #2}%
    \fi
    \null
  }%
}}
\def\inleftmargin{\doinmargin l}
\def\inrightmargin{\doinmargin r}
%
% @inmargin{TEXT [, RIGHT-TEXT]}
% (if RIGHT-TEXT is given, use TEXT for left page, RIGHT-TEXT for right;
% else use TEXT for both).
%
\def\inmargin#1{\parseinmargin #1,,\finish}
\def\parseinmargin#1,#2,#3\finish{% not perfect, but better than nothing.
  \setbox0 = \hbox{\ignorespaces #2}%
  \ifdim\wd0 > 0pt
    \def\lefttext{#1}%  have both texts
    \def\righttext{#2}%
  \else
    \def\lefttext{#1}%  have only one text
    \def\righttext{#1}%
  \fi
  %
  \ifodd\pageno
    \def\temp{\inrightmargin\righttext}% odd page -> outside is right margin
  \else
    \def\temp{\inleftmargin\lefttext}%
  \fi
  \temp
}

% @| inserts a changebar to the left of the current line.  It should
% surround any changed text.  This approach does *not* work if the
% change spans more than two lines of output.  To handle that, we would
% have adopt a much more difficult approach (putting marks into the main
% vertical list for the beginning and end of each change).  This command
% is not documented, not supported, and doesn't work.
%
\def\|{%
  % \vadjust can only be used in horizontal mode.
  \leavevmode
  %
  % Append this vertical mode material after the current line in the output.
  \vadjust{%
    % We want to insert a rule with the height and depth of the current
    % leading; that is exactly what \strutbox is supposed to record.
    \vskip-\baselineskip
    %
    % \vadjust-items are inserted at the left edge of the type.  So
    % the \llap here moves out into the left-hand margin.
    \llap{%
      %
      % For a thicker or thinner bar, change the `1pt'.
      \vrule height\baselineskip width1pt
      %
      % This is the space between the bar and the text.
      \hskip 12pt
    }%
  }%
}

% @include FILE -- \input text of FILE.
%
\def\include{\parseargusing\filenamecatcodes\includezzz}
\def\includezzz#1{%
  \pushthisfilestack
  \def\thisfile{#1}%
  {%
    \makevalueexpandable  % we want to expand any @value in FILE.
    \turnoffactive        % and allow special characters in the expansion
    \indexnofonts         % Allow `@@' and other weird things in file names.
    \wlog{texinfo.tex: doing @include of #1^^J}%
    \edef\temp{\noexpand\input #1 }%
    %
    % This trickery is to read FILE outside of a group, in case it makes
    % definitions, etc.
    \expandafter
  }\temp
  \popthisfilestack
}
\def\filenamecatcodes{%
  \catcode`\\=\other
  \catcode`~=\other
  \catcode`^=\other
  \catcode`_=\other
  \catcode`|=\other
  \catcode`<=\other
  \catcode`>=\other
  \catcode`+=\other
  \catcode`-=\other
  \catcode`\`=\other
  \catcode`\'=\other
}

\def\pushthisfilestack{%
  \expandafter\pushthisfilestackX\popthisfilestack\StackTerm
}
\def\pushthisfilestackX{%
  \expandafter\pushthisfilestackY\thisfile\StackTerm
}
\def\pushthisfilestackY #1\StackTerm #2\StackTerm {%
  \gdef\popthisfilestack{\gdef\thisfile{#1}\gdef\popthisfilestack{#2}}%
}

\def\popthisfilestack{\errthisfilestackempty}
\def\errthisfilestackempty{\errmessage{Internal error:
  the stack of filenames is empty.}}
%
\def\thisfile{}

% @center line
% outputs that line, centered.
%
\parseargdef\center{%
  \ifhmode
    \let\centersub\centerH
  \else
    \let\centersub\centerV
  \fi
  \centersub{\hfil \ignorespaces#1\unskip \hfil}%
  \let\centersub\relax % don't let the definition persist, just in case
}
\def\centerH#1{{%
  \hfil\break
  \advance\hsize by -\leftskip
  \advance\hsize by -\rightskip
  \line{#1}%
  \break
}}
%
\newcount\centerpenalty
\def\centerV#1{%
  % The idea here is the same as in \startdefun, \cartouche, etc.: if
  % @center is the first thing after a section heading, we need to wipe
  % out the negative parskip inserted by \sectionheading, but still
  % prevent a page break here.
  \centerpenalty = \lastpenalty
  \ifnum\centerpenalty>10000 \vskip\parskip \fi
  \ifnum\centerpenalty>9999 \penalty\centerpenalty \fi
  \line{\kern\leftskip #1\kern\rightskip}%
}

% @sp n   outputs n lines of vertical space
%
\parseargdef\sp{\vskip #1\baselineskip}

% @comment ...line which is ignored...
% @c is the same as @comment
% @ignore ... @end ignore  is another way to write a comment
%
\def\comment{\begingroup \catcode`\^^M=\other%
\catcode`\@=\other \catcode`\{=\other \catcode`\}=\other%
\commentxxx}
{\catcode`\^^M=\other \gdef\commentxxx#1^^M{\endgroup}}
%
\let\c=\comment

% @paragraphindent NCHARS
% We'll use ems for NCHARS, close enough.
% NCHARS can also be the word `asis' or `none'.
% We cannot feasibly implement @paragraphindent asis, though.
%
\def\asisword{asis} % no translation, these are keywords
\def\noneword{none}
%
\parseargdef\paragraphindent{%
  \def\temp{#1}%
  \ifx\temp\asisword
  \else
    \ifx\temp\noneword
      \defaultparindent = 0pt
    \else
      \defaultparindent = #1em
    \fi
  \fi
  \parindent = \defaultparindent
}

% @exampleindent NCHARS
% We'll use ems for NCHARS like @paragraphindent.
% It seems @exampleindent asis isn't necessary, but
% I preserve it to make it similar to @paragraphindent.
\parseargdef\exampleindent{%
  \def\temp{#1}%
  \ifx\temp\asisword
  \else
    \ifx\temp\noneword
      \lispnarrowing = 0pt
    \else
      \lispnarrowing = #1em
    \fi
  \fi
}

% @firstparagraphindent WORD
% If WORD is `none', then suppress indentation of the first paragraph
% after a section heading.  If WORD is `insert', then do indent at such
% paragraphs.
%
% The paragraph indentation is suppressed or not by calling
% \suppressfirstparagraphindent, which the sectioning commands do.
% We switch the definition of this back and forth according to WORD.
% By default, we suppress indentation.
%
\def\suppressfirstparagraphindent{\dosuppressfirstparagraphindent}
\def\insertword{insert}
%
\parseargdef\firstparagraphindent{%
  \def\temp{#1}%
  \ifx\temp\noneword
    \let\suppressfirstparagraphindent = \dosuppressfirstparagraphindent
  \else\ifx\temp\insertword
    \let\suppressfirstparagraphindent = \relax
  \else
    \errhelp = \EMsimple
    \errmessage{Unknown @firstparagraphindent option `\temp'}%
  \fi\fi
}

% Here is how we actually suppress indentation.  Redefine \everypar to
% \kern backwards by \parindent, and then reset itself to empty.
%
% We also make \indent itself not actually do anything until the next
% paragraph.
%
\gdef\dosuppressfirstparagraphindent{%
  \gdef\indent{%
    \restorefirstparagraphindent
    \indent
  }%
  \gdef\noindent{%
    \restorefirstparagraphindent
    \noindent
  }%
  \global\everypar = {%
    \kern -\parindent
    \restorefirstparagraphindent
  }%
}

\gdef\restorefirstparagraphindent{%
  \global \let \indent = \ptexindent
  \global \let \noindent = \ptexnoindent
  \global \everypar = {}%
}


% @refill is a no-op.
\let\refill=\relax

% If working on a large document in chapters, it is convenient to
% be able to disable indexing, cross-referencing, and contents, for test runs.
% This is done with @novalidate (before @setfilename).
%
\newif\iflinks \linkstrue % by default we want the aux files.
\let\novalidate = \linksfalse

% @setfilename is done at the beginning of every texinfo file.
% So open here the files we need to have open while reading the input.
% This makes it possible to make a .fmt file for texinfo.
\def\setfilename{%
   \fixbackslash  % Turn off hack to swallow `\input texinfo'.
   \iflinks
     \tryauxfile
     % Open the new aux file.  TeX will close it automatically at exit.
     \immediate\openout\auxfile=\jobname.aux
   \fi % \openindices needs to do some work in any case.
   \openindices
   \let\setfilename=\comment % Ignore extra @setfilename cmds.
   %
   % If texinfo.cnf is present on the system, read it.
   % Useful for site-wide @afourpaper, etc.
   \openin 1 texinfo.cnf
   \ifeof 1 \else \input texinfo.cnf \fi
   \closein 1
   %
   \comment % Ignore the actual filename.
}

% Called from \setfilename.
%
\def\openindices{%
  \newindex{cp}%
  \newcodeindex{fn}%
  \newcodeindex{vr}%
  \newcodeindex{tp}%
  \newcodeindex{ky}%
  \newcodeindex{pg}%
}

% @bye.
\outer\def\bye{\pagealignmacro\tracingstats=1\ptexend}


\message{pdf,}
% adobe `portable' document format
\newcount\tempnum
\newcount\lnkcount
\newtoks\filename
\newcount\filenamelength
\newcount\pgn
\newtoks\toksA
\newtoks\toksB
\newtoks\toksC
\newtoks\toksD
\newbox\boxA
\newcount\countA
\newif\ifpdf
\newif\ifpdfmakepagedest

% when pdftex is run in dvi mode, \pdfoutput is defined (so \pdfoutput=1
% can be set).  So we test for \relax and 0 as well as being undefined.
\ifx\pdfoutput\thisisundefined
\else
  \ifx\pdfoutput\relax
  \else
    \ifcase\pdfoutput
    \else
      \pdftrue
    \fi
  \fi
\fi

% PDF uses PostScript string constants for the names of xref targets,
% for display in the outlines, and in other places.  Thus, we have to
% double any backslashes.  Otherwise, a name like "\node" will be
% interpreted as a newline (\n), followed by o, d, e.  Not good.
% 
% See http://www.ntg.nl/pipermail/ntg-pdftex/2004-July/000654.html and
% related messages.  The final outcome is that it is up to the TeX user
% to double the backslashes and otherwise make the string valid, so
% that's what we do.  pdftex 1.30.0 (ca.2005) introduced a primitive to
% do this reliably, so we use it.

% #1 is a control sequence in which to do the replacements,
% which we \xdef.
\def\txiescapepdf#1{%
  \ifx\pdfescapestring\thisisundefined
    % No primitive available; should we give a warning or log?
    % Many times it won't matter.
  \else
    % The expandable \pdfescapestring primitive escapes parentheses,
    % backslashes, and other special chars.
    \xdef#1{\pdfescapestring{#1}}%
  \fi
}

\newhelp\nopdfimagehelp{Texinfo supports .png, .jpg, .jpeg, and .pdf images
with PDF output, and none of those formats could be found.  (.eps cannot
be supported due to the design of the PDF format; use regular TeX (DVI
output) for that.)}

\ifpdf
  %
  % Color manipulation macros based on pdfcolor.tex,
  % except using rgb instead of cmyk; the latter is said to render as a
  % very dark gray on-screen and a very dark halftone in print, instead
  % of actual black.
  \def\rgbDarkRed{0.50 0.09 0.12}
  \def\rgbBlack{0 0 0}
  %
  % k sets the color for filling (usual text, etc.);
  % K sets the color for stroking (thin rules, e.g., normal _'s).
  \def\pdfsetcolor#1{\pdfliteral{#1 rg  #1 RG}}
  %
  % Set color, and create a mark which defines \thiscolor accordingly,
  % so that \makeheadline knows which color to restore.
  \def\setcolor#1{%
    \xdef\lastcolordefs{\gdef\noexpand\thiscolor{#1}}%
    \domark
    \pdfsetcolor{#1}%
  }
  %
  \def\maincolor{\rgbBlack}
  \pdfsetcolor{\maincolor}
  \edef\thiscolor{\maincolor}
  \def\lastcolordefs{}
  %
  \def\makefootline{%
    \baselineskip24pt
    \line{\pdfsetcolor{\maincolor}\the\footline}%
  }
  %
  \def\makeheadline{%
    \vbox to 0pt{%
      \vskip-22.5pt
      \line{%
        \vbox to8.5pt{}%
        % Extract \thiscolor definition from the marks.
        \getcolormarks
        % Typeset the headline with \maincolor, then restore the color.
        \pdfsetcolor{\maincolor}\the\headline\pdfsetcolor{\thiscolor}%
      }%
      \vss
    }%
    \nointerlineskip
  }
  %
  %
  \pdfcatalog{/PageMode /UseOutlines}
  %
  % #1 is image name, #2 width (might be empty/whitespace), #3 height (ditto).
  \def\dopdfimage#1#2#3{%
    \def\pdfimagewidth{#2}\setbox0 = \hbox{\ignorespaces #2}%
    \def\pdfimageheight{#3}\setbox2 = \hbox{\ignorespaces #3}%
    %
    % pdftex (and the PDF format) support .pdf, .png, .jpg (among
    % others).  Let's try in that order, PDF first since if
    % someone has a scalable image, presumably better to use that than a
    % bitmap.
    \let\pdfimgext=\empty
    \begingroup
      \openin 1 #1.pdf \ifeof 1
        \openin 1 #1.PDF \ifeof 1
          \openin 1 #1.png \ifeof 1
            \openin 1 #1.jpg \ifeof 1
              \openin 1 #1.jpeg \ifeof 1
                \openin 1 #1.JPG \ifeof 1
                  \errhelp = \nopdfimagehelp
                  \errmessage{Could not find image file #1 for pdf}%
                \else \gdef\pdfimgext{JPG}%
                \fi
              \else \gdef\pdfimgext{jpeg}%
              \fi
            \else \gdef\pdfimgext{jpg}%
            \fi
          \else \gdef\pdfimgext{png}%
          \fi
        \else \gdef\pdfimgext{PDF}%
        \fi
      \else \gdef\pdfimgext{pdf}%
      \fi
      \closein 1
    \endgroup
    %
    % without \immediate, ancient pdftex seg faults when the same image is
    % included twice.  (Version 3.14159-pre-1.0-unofficial-20010704.)
    \ifnum\pdftexversion < 14
      \immediate\pdfimage
    \else
      \immediate\pdfximage
    \fi
      \ifdim \wd0 >0pt width \pdfimagewidth \fi
      \ifdim \wd2 >0pt height \pdfimageheight \fi
      \ifnum\pdftexversion<13
         #1.\pdfimgext
       \else
         {#1.\pdfimgext}%
       \fi
    \ifnum\pdftexversion < 14 \else
      \pdfrefximage \pdflastximage
    \fi}
  %
  \def\pdfmkdest#1{{%
    % We have to set dummies so commands such as @code, and characters
    % such as \, aren't expanded when present in a section title.
    \indexnofonts
    \turnoffactive
    \makevalueexpandable
    \def\pdfdestname{#1}%
    \txiescapepdf\pdfdestname
    \safewhatsit{\pdfdest name{\pdfdestname} xyz}%
  }}
  %
  % used to mark target names; must be expandable.
  \def\pdfmkpgn#1{#1}
  %
  % by default, use a color that is dark enough to print on paper as
  % nearly black, but still distinguishable for online viewing.
  \def\urlcolor{\rgbDarkRed}
  \def\linkcolor{\rgbDarkRed}
  \def\endlink{\setcolor{\maincolor}\pdfendlink}
  %
  % Adding outlines to PDF; macros for calculating structure of outlines
  % come from Petr Olsak
  \def\expnumber#1{\expandafter\ifx\csname#1\endcsname\relax 0%
    \else \csname#1\endcsname \fi}
  \def\advancenumber#1{\tempnum=\expnumber{#1}\relax
    \advance\tempnum by 1
    \expandafter\xdef\csname#1\endcsname{\the\tempnum}}
  %
  % #1 is the section text, which is what will be displayed in the
  % outline by the pdf viewer.  #2 is the pdf expression for the number
  % of subentries (or empty, for subsubsections).  #3 is the node text,
  % which might be empty if this toc entry had no corresponding node.
  % #4 is the page number
  %
  \def\dopdfoutline#1#2#3#4{%
    % Generate a link to the node text if that exists; else, use the
    % page number.  We could generate a destination for the section
    % text in the case where a section has no node, but it doesn't
    % seem worth the trouble, since most documents are normally structured.
    \edef\pdfoutlinedest{#3}%
    \ifx\pdfoutlinedest\empty
      \def\pdfoutlinedest{#4}%
    \else
      \txiescapepdf\pdfoutlinedest
    \fi
    %
    % Also escape PDF chars in the display string.
    \edef\pdfoutlinetext{#1}%
    \txiescapepdf\pdfoutlinetext
    %
    \pdfoutline goto name{\pdfmkpgn{\pdfoutlinedest}}#2{\pdfoutlinetext}%
  }
  %
  \def\pdfmakeoutlines{%
    \begingroup
      % Read toc silently, to get counts of subentries for \pdfoutline.
      \def\partentry##1##2##3##4{}% ignore parts in the outlines
      \def\numchapentry##1##2##3##4{%
	\def\thischapnum{##2}%
	\def\thissecnum{0}%
	\def\thissubsecnum{0}%
      }%
      \def\numsecentry##1##2##3##4{%
	\advancenumber{chap\thischapnum}%
	\def\thissecnum{##2}%
	\def\thissubsecnum{0}%
      }%
      \def\numsubsecentry##1##2##3##4{%
	\advancenumber{sec\thissecnum}%
	\def\thissubsecnum{##2}%
      }%
      \def\numsubsubsecentry##1##2##3##4{%
	\advancenumber{subsec\thissubsecnum}%
      }%
      \def\thischapnum{0}%
      \def\thissecnum{0}%
      \def\thissubsecnum{0}%
      %
      % use \def rather than \let here because we redefine \chapentry et
      % al. a second time, below.
      \def\appentry{\numchapentry}%
      \def\appsecentry{\numsecentry}%
      \def\appsubsecentry{\numsubsecentry}%
      \def\appsubsubsecentry{\numsubsubsecentry}%
      \def\unnchapentry{\numchapentry}%
      \def\unnsecentry{\numsecentry}%
      \def\unnsubsecentry{\numsubsecentry}%
      \def\unnsubsubsecentry{\numsubsubsecentry}%
      \readdatafile{toc}%
      %
      % Read toc second time, this time actually producing the outlines.
      % The `-' means take the \expnumber as the absolute number of
      % subentries, which we calculated on our first read of the .toc above.
      %
      % We use the node names as the destinations.
      \def\numchapentry##1##2##3##4{%
        \dopdfoutline{##1}{count-\expnumber{chap##2}}{##3}{##4}}%
      \def\numsecentry##1##2##3##4{%
        \dopdfoutline{##1}{count-\expnumber{sec##2}}{##3}{##4}}%
      \def\numsubsecentry##1##2##3##4{%
        \dopdfoutline{##1}{count-\expnumber{subsec##2}}{##3}{##4}}%
      \def\numsubsubsecentry##1##2##3##4{% count is always zero
        \dopdfoutline{##1}{}{##3}{##4}}%
      %
      % PDF outlines are displayed using system fonts, instead of
      % document fonts.  Therefore we cannot use special characters,
      % since the encoding is unknown.  For example, the eogonek from
      % Latin 2 (0xea) gets translated to a | character.  Info from
      % Staszek Wawrykiewicz, 19 Jan 2004 04:09:24 +0100.
      %
      % TODO this right, we have to translate 8-bit characters to
      % their "best" equivalent, based on the @documentencoding.  Too
      % much work for too little return.  Just use the ASCII equivalents
      % we use for the index sort strings.
      % 
      \indexnofonts
      \setupdatafile
      % We can have normal brace characters in the PDF outlines, unlike
      % Texinfo index files.  So set that up.
      \def\{{\lbracecharliteral}%
      \def\}{\rbracecharliteral}%
      \catcode`\\=\active \otherbackslash
      \input \tocreadfilename
    \endgroup
  }
  {\catcode`[=1 \catcode`]=2
   \catcode`{=\other \catcode`}=\other
   \gdef\lbracecharliteral[{]%
   \gdef\rbracecharliteral[}]%
  ]
  %
  \def\skipspaces#1{\def\PP{#1}\def\D{|}%
    \ifx\PP\D\let\nextsp\relax
    \else\let\nextsp\skipspaces
      \addtokens{\filename}{\PP}%
      \advance\filenamelength by 1
    \fi
    \nextsp}
  \def\getfilename#1{%
    \filenamelength=0
    % If we don't expand the argument now, \skipspaces will get
    % snagged on things like "@value{foo}".
    \edef\temp{#1}%
    \expandafter\skipspaces\temp|\relax
  }
  \ifnum\pdftexversion < 14
    \let \startlink \pdfannotlink
  \else
    \let \startlink \pdfstartlink
  \fi
  % make a live url in pdf output.
  \def\pdfurl#1{%
    \begingroup
      % it seems we really need yet another set of dummies; have not
      % tried to figure out what each command should do in the context
      % of @url.  for now, just make @/ a no-op, that's the only one
      % people have actually reported a problem with.
      %
      \normalturnoffactive
      \def\@{@}%
      \let\/=\empty
      \makevalueexpandable
      % do we want to go so far as to use \indexnofonts instead of just
      % special-casing \var here?
      \def\var##1{##1}%
      %
      \leavevmode\setcolor{\urlcolor}%
      \startlink attr{/Border [0 0 0]}%
        user{/Subtype /Link /A << /S /URI /URI (#1) >>}%
    \endgroup}
  \def\pdfgettoks#1.{\setbox\boxA=\hbox{\toksA={#1.}\toksB={}\maketoks}}
  \def\addtokens#1#2{\edef\addtoks{\noexpand#1={\the#1#2}}\addtoks}
  \def\adn#1{\addtokens{\toksC}{#1}\global\countA=1\let\next=\maketoks}
  \def\poptoks#1#2|ENDTOKS|{\let\first=#1\toksD={#1}\toksA={#2}}
  \def\maketoks{%
    \expandafter\poptoks\the\toksA|ENDTOKS|\relax
    \ifx\first0\adn0
    \else\ifx\first1\adn1 \else\ifx\first2\adn2 \else\ifx\first3\adn3
    \else\ifx\first4\adn4 \else\ifx\first5\adn5 \else\ifx\first6\adn6
    \else\ifx\first7\adn7 \else\ifx\first8\adn8 \else\ifx\first9\adn9
    \else
      \ifnum0=\countA\else\makelink\fi
      \ifx\first.\let\next=\done\else
        \let\next=\maketoks
        \addtokens{\toksB}{\the\toksD}
        \ifx\first,\addtokens{\toksB}{\space}\fi
      \fi
    \fi\fi\fi\fi\fi\fi\fi\fi\fi\fi
    \next}
  \def\makelink{\addtokens{\toksB}%
    {\noexpand\pdflink{\the\toksC}}\toksC={}\global\countA=0}
  \def\pdflink#1{%
    \startlink attr{/Border [0 0 0]} goto name{\pdfmkpgn{#1}}
    \setcolor{\linkcolor}#1\endlink}
  \def\done{\edef\st{\global\noexpand\toksA={\the\toksB}}\st}
\else
  % non-pdf mode
  \let\pdfmkdest = \gobble
  \let\pdfurl = \gobble
  \let\endlink = \relax
  \let\setcolor = \gobble
  \let\pdfsetcolor = \gobble
  \let\pdfmakeoutlines = \relax
\fi  % \ifx\pdfoutput


\message{fonts,}

% Change the current font style to #1, remembering it in \curfontstyle.
% For now, we do not accumulate font styles: @b{@i{foo}} prints foo in
% italics, not bold italics.
%
\def\setfontstyle#1{%
  \def\curfontstyle{#1}% not as a control sequence, because we are \edef'd.
  \csname ten#1\endcsname  % change the current font
}

% Select #1 fonts with the current style.
%
\def\selectfonts#1{\csname #1fonts\endcsname \csname\curfontstyle\endcsname}

\def\rm{\fam=0 \setfontstyle{rm}}
\def\it{\fam=\itfam \setfontstyle{it}}
\def\sl{\fam=\slfam \setfontstyle{sl}}
\def\bf{\fam=\bffam \setfontstyle{bf}}\def\bfstylename{bf}
\def\tt{\fam=\ttfam \setfontstyle{tt}}

% Unfortunately, we have to override this for titles and the like, since
% in those cases "rm" is bold.  Sigh.
\def\rmisbold{\rm\def\curfontstyle{bf}}

% Texinfo sort of supports the sans serif font style, which plain TeX does not.
% So we set up a \sf.
\newfam\sffam
\def\sf{\fam=\sffam \setfontstyle{sf}}
\let\li = \sf % Sometimes we call it \li, not \sf.

% We don't need math for this font style.
\def\ttsl{\setfontstyle{ttsl}}


% Set the baselineskip to #1, and the lineskip and strut size
% correspondingly.  There is no deep meaning behind these magic numbers
% used as factors; they just match (closely enough) what Knuth defined.
%
\def\lineskipfactor{.08333}
\def\strutheightpercent{.70833}
\def\strutdepthpercent {.29167}
%
% can get a sort of poor man's double spacing by redefining this.
\def\baselinefactor{1}
%
\newdimen\textleading
\def\setleading#1{%
  \dimen0 = #1\relax
  \normalbaselineskip = \baselinefactor\dimen0
  \normallineskip = \lineskipfactor\normalbaselineskip
  \normalbaselines
  \setbox\strutbox =\hbox{%
    \vrule width0pt height\strutheightpercent\baselineskip
                    depth \strutdepthpercent \baselineskip
  }%
}

% PDF CMaps.  See also LaTeX's t1.cmap.
%
% do nothing with this by default.
\expandafter\let\csname cmapOT1\endcsname\gobble
\expandafter\let\csname cmapOT1IT\endcsname\gobble
\expandafter\let\csname cmapOT1TT\endcsname\gobble

% if we are producing pdf, and we have \pdffontattr, then define cmaps.
% (\pdffontattr was introduced many years ago, but people still run
% older pdftex's; it's easy to conditionalize, so we do.)
\ifpdf \ifx\pdffontattr\thisisundefined \else
  \begingroup
    \catcode`\^^M=\active \def^^M{^^J}% Output line endings as the ^^J char.
    \catcode`\%=12 \immediate\pdfobj stream {%!PS-Adobe-3.0 Resource-CMap
%%DocumentNeededResources: ProcSet (CIDInit)
%%IncludeResource: ProcSet (CIDInit)
%%BeginResource: CMap (TeX-OT1-0)
%%Title: (TeX-OT1-0 TeX OT1 0)
%%Version: 1.000
%%EndComments
/CIDInit /ProcSet findresource begin
12 dict begin
begincmap
/CIDSystemInfo
<< /Registry (TeX)
/Ordering (OT1)
/Supplement 0
>> def
/CMapName /TeX-OT1-0 def
/CMapType 2 def
1 begincodespacerange
<00> <7F>
endcodespacerange
8 beginbfrange
<00> <01> <0393>
<09> <0A> <03A8>
<23> <26> <0023>
<28> <3B> <0028>
<3F> <5B> <003F>
<5D> <5E> <005D>
<61> <7A> <0061>
<7B> <7C> <2013>
endbfrange
40 beginbfchar
<02> <0398>
<03> <039B>
<04> <039E>
<05> <03A0>
<06> <03A3>
<07> <03D2>
<08> <03A6>
<0B> <00660066>
<0C> <00660069>
<0D> <0066006C>
<0E> <006600660069>
<0F> <00660066006C>
<10> <0131>
<11> <0237>
<12> <0060>
<13> <00B4>
<14> <02C7>
<15> <02D8>
<16> <00AF>
<17> <02DA>
<18> <00B8>
<19> <00DF>
<1A> <00E6>
<1B> <0153>
<1C> <00F8>
<1D> <00C6>
<1E> <0152>
<1F> <00D8>
<21> <0021>
<22> <201D>
<27> <2019>
<3C> <00A1>
<3D> <003D>
<3E> <00BF>
<5C> <201C>
<5F> <02D9>
<60> <2018>
<7D> <02DD>
<7E> <007E>
<7F> <00A8>
endbfchar
endcmap
CMapName currentdict /CMap defineresource pop
end
end
%%EndResource
%%EOF
    }\endgroup
  \expandafter\edef\csname cmapOT1\endcsname#1{%
    \pdffontattr#1{/ToUnicode \the\pdflastobj\space 0 R}%
  }%
%
% \cmapOT1IT
  \begingroup
    \catcode`\^^M=\active \def^^M{^^J}% Output line endings as the ^^J char.
    \catcode`\%=12 \immediate\pdfobj stream {%!PS-Adobe-3.0 Resource-CMap
%%DocumentNeededResources: ProcSet (CIDInit)
%%IncludeResource: ProcSet (CIDInit)
%%BeginResource: CMap (TeX-OT1IT-0)
%%Title: (TeX-OT1IT-0 TeX OT1IT 0)
%%Version: 1.000
%%EndComments
/CIDInit /ProcSet findresource begin
12 dict begin
begincmap
/CIDSystemInfo
<< /Registry (TeX)
/Ordering (OT1IT)
/Supplement 0
>> def
/CMapName /TeX-OT1IT-0 def
/CMapType 2 def
1 begincodespacerange
<00> <7F>
endcodespacerange
8 beginbfrange
<00> <01> <0393>
<09> <0A> <03A8>
<25> <26> <0025>
<28> <3B> <0028>
<3F> <5B> <003F>
<5D> <5E> <005D>
<61> <7A> <0061>
<7B> <7C> <2013>
endbfrange
42 beginbfchar
<02> <0398>
<03> <039B>
<04> <039E>
<05> <03A0>
<06> <03A3>
<07> <03D2>
<08> <03A6>
<0B> <00660066>
<0C> <00660069>
<0D> <0066006C>
<0E> <006600660069>
<0F> <00660066006C>
<10> <0131>
<11> <0237>
<12> <0060>
<13> <00B4>
<14> <02C7>
<15> <02D8>
<16> <00AF>
<17> <02DA>
<18> <00B8>
<19> <00DF>
<1A> <00E6>
<1B> <0153>
<1C> <00F8>
<1D> <00C6>
<1E> <0152>
<1F> <00D8>
<21> <0021>
<22> <201D>
<23> <0023>
<24> <00A3>
<27> <2019>
<3C> <00A1>
<3D> <003D>
<3E> <00BF>
<5C> <201C>
<5F> <02D9>
<60> <2018>
<7D> <02DD>
<7E> <007E>
<7F> <00A8>
endbfchar
endcmap
CMapName currentdict /CMap defineresource pop
end
end
%%EndResource
%%EOF
    }\endgroup
  \expandafter\edef\csname cmapOT1IT\endcsname#1{%
    \pdffontattr#1{/ToUnicode \the\pdflastobj\space 0 R}%
  }%
%
% \cmapOT1TT
  \begingroup
    \catcode`\^^M=\active \def^^M{^^J}% Output line endings as the ^^J char.
    \catcode`\%=12 \immediate\pdfobj stream {%!PS-Adobe-3.0 Resource-CMap
%%DocumentNeededResources: ProcSet (CIDInit)
%%IncludeResource: ProcSet (CIDInit)
%%BeginResource: CMap (TeX-OT1TT-0)
%%Title: (TeX-OT1TT-0 TeX OT1TT 0)
%%Version: 1.000
%%EndComments
/CIDInit /ProcSet findresource begin
12 dict begin
begincmap
/CIDSystemInfo
<< /Registry (TeX)
/Ordering (OT1TT)
/Supplement 0
>> def
/CMapName /TeX-OT1TT-0 def
/CMapType 2 def
1 begincodespacerange
<00> <7F>
endcodespacerange
5 beginbfrange
<00> <01> <0393>
<09> <0A> <03A8>
<21> <26> <0021>
<28> <5F> <0028>
<61> <7E> <0061>
endbfrange
32 beginbfchar
<02> <0398>
<03> <039B>
<04> <039E>
<05> <03A0>
<06> <03A3>
<07> <03D2>
<08> <03A6>
<0B> <2191>
<0C> <2193>
<0D> <0027>
<0E> <00A1>
<0F> <00BF>
<10> <0131>
<11> <0237>
<12> <0060>
<13> <00B4>
<14> <02C7>
<15> <02D8>
<16> <00AF>
<17> <02DA>
<18> <00B8>
<19> <00DF>
<1A> <00E6>
<1B> <0153>
<1C> <00F8>
<1D> <00C6>
<1E> <0152>
<1F> <00D8>
<20> <2423>
<27> <2019>
<60> <2018>
<7F> <00A8>
endbfchar
endcmap
CMapName currentdict /CMap defineresource pop
end
end
%%EndResource
%%EOF
    }\endgroup
  \expandafter\edef\csname cmapOT1TT\endcsname#1{%
    \pdffontattr#1{/ToUnicode \the\pdflastobj\space 0 R}%
  }%
\fi\fi


% Set the font macro #1 to the font named \fontprefix#2.
% #3 is the font's design size, #4 is a scale factor, #5 is the CMap
% encoding (only OT1, OT1IT and OT1TT are allowed, or empty to omit).
% Example:
% #1 = \textrm
% #2 = \rmshape
% #3 = 10
% #4 = \mainmagstep
% #5 = OT1
%
\def\setfont#1#2#3#4#5{%
  \font#1=\fontprefix#2#3 scaled #4
  \csname cmap#5\endcsname#1%
}
% This is what gets called when #5 of \setfont is empty.
\let\cmap\gobble
%
% (end of cmaps)

% Use cm as the default font prefix.
% To specify the font prefix, you must define \fontprefix
% before you read in texinfo.tex.
\ifx\fontprefix\thisisundefined
\def\fontprefix{cm}
\fi
% Support font families that don't use the same naming scheme as CM.
\def\rmshape{r}
\def\rmbshape{bx}               % where the normal face is bold
\def\bfshape{b}
\def\bxshape{bx}
\def\ttshape{tt}
\def\ttbshape{tt}
\def\ttslshape{sltt}
\def\itshape{ti}
\def\itbshape{bxti}
\def\slshape{sl}
\def\slbshape{bxsl}
\def\sfshape{ss}
\def\sfbshape{ss}
\def\scshape{csc}
\def\scbshape{csc}

% Definitions for a main text size of 11pt.  (The default in Texinfo.)
%
\def\definetextfontsizexi{%
% Text fonts (11.2pt, magstep1).
\def\textnominalsize{11pt}
\edef\mainmagstep{\magstephalf}
\setfont\textrm\rmshape{10}{\mainmagstep}{OT1}
\setfont\texttt\ttshape{10}{\mainmagstep}{OT1TT}
\setfont\textbf\bfshape{10}{\mainmagstep}{OT1}
\setfont\textit\itshape{10}{\mainmagstep}{OT1IT}
\setfont\textsl\slshape{10}{\mainmagstep}{OT1}
\setfont\textsf\sfshape{10}{\mainmagstep}{OT1}
\setfont\textsc\scshape{10}{\mainmagstep}{OT1}
\setfont\textttsl\ttslshape{10}{\mainmagstep}{OT1TT}
\font\texti=cmmi10 scaled \mainmagstep
\font\textsy=cmsy10 scaled \mainmagstep
\def\textecsize{1095}

% A few fonts for @defun names and args.
\setfont\defbf\bfshape{10}{\magstep1}{OT1}
\setfont\deftt\ttshape{10}{\magstep1}{OT1TT}
\setfont\defttsl\ttslshape{10}{\magstep1}{OT1TT}
\def\df{\let\tentt=\deftt \let\tenbf = \defbf \let\tenttsl=\defttsl \bf}

% Fonts for indices, footnotes, small examples (9pt).
\def\smallnominalsize{9pt}
\setfont\smallrm\rmshape{9}{1000}{OT1}
\setfont\smalltt\ttshape{9}{1000}{OT1TT}
\setfont\smallbf\bfshape{10}{900}{OT1}
\setfont\smallit\itshape{9}{1000}{OT1IT}
\setfont\smallsl\slshape{9}{1000}{OT1}
\setfont\smallsf\sfshape{9}{1000}{OT1}
\setfont\smallsc\scshape{10}{900}{OT1}
\setfont\smallttsl\ttslshape{10}{900}{OT1TT}
\font\smalli=cmmi9
\font\smallsy=cmsy9
\def\smallecsize{0900}

% Fonts for small examples (8pt).
\def\smallernominalsize{8pt}
\setfont\smallerrm\rmshape{8}{1000}{OT1}
\setfont\smallertt\ttshape{8}{1000}{OT1TT}
\setfont\smallerbf\bfshape{10}{800}{OT1}
\setfont\smallerit\itshape{8}{1000}{OT1IT}
\setfont\smallersl\slshape{8}{1000}{OT1}
\setfont\smallersf\sfshape{8}{1000}{OT1}
\setfont\smallersc\scshape{10}{800}{OT1}
\setfont\smallerttsl\ttslshape{10}{800}{OT1TT}
\font\smalleri=cmmi8
\font\smallersy=cmsy8
\def\smallerecsize{0800}

% Fonts for title page (20.4pt):
\def\titlenominalsize{20pt}
\setfont\titlerm\rmbshape{12}{\magstep3}{OT1}
\setfont\titleit\itbshape{10}{\magstep4}{OT1IT}
\setfont\titlesl\slbshape{10}{\magstep4}{OT1}
\setfont\titlett\ttbshape{12}{\magstep3}{OT1TT}
\setfont\titlettsl\ttslshape{10}{\magstep4}{OT1TT}
\setfont\titlesf\sfbshape{17}{\magstep1}{OT1}
\let\titlebf=\titlerm
\setfont\titlesc\scbshape{10}{\magstep4}{OT1}
\font\titlei=cmmi12 scaled \magstep3
\font\titlesy=cmsy10 scaled \magstep4
\def\titleecsize{2074}

% Chapter (and unnumbered) fonts (17.28pt).
\def\chapnominalsize{17pt}
\setfont\chaprm\rmbshape{12}{\magstep2}{OT1}
\setfont\chapit\itbshape{10}{\magstep3}{OT1IT}
\setfont\chapsl\slbshape{10}{\magstep3}{OT1}
\setfont\chaptt\ttbshape{12}{\magstep2}{OT1TT}
\setfont\chapttsl\ttslshape{10}{\magstep3}{OT1TT}
\setfont\chapsf\sfbshape{17}{1000}{OT1}
\let\chapbf=\chaprm
\setfont\chapsc\scbshape{10}{\magstep3}{OT1}
\font\chapi=cmmi12 scaled \magstep2
\font\chapsy=cmsy10 scaled \magstep3
\def\chapecsize{1728}

% Section fonts (14.4pt).
\def\secnominalsize{14pt}
\setfont\secrm\rmbshape{12}{\magstep1}{OT1}
\setfont\secit\itbshape{10}{\magstep2}{OT1IT}
\setfont\secsl\slbshape{10}{\magstep2}{OT1}
\setfont\sectt\ttbshape{12}{\magstep1}{OT1TT}
\setfont\secttsl\ttslshape{10}{\magstep2}{OT1TT}
\setfont\secsf\sfbshape{12}{\magstep1}{OT1}
\let\secbf\secrm
\setfont\secsc\scbshape{10}{\magstep2}{OT1}
\font\seci=cmmi12 scaled \magstep1
\font\secsy=cmsy10 scaled \magstep2
\def\sececsize{1440}

% Subsection fonts (13.15pt).
\def\ssecnominalsize{13pt}
\setfont\ssecrm\rmbshape{12}{\magstephalf}{OT1}
\setfont\ssecit\itbshape{10}{1315}{OT1IT}
\setfont\ssecsl\slbshape{10}{1315}{OT1}
\setfont\ssectt\ttbshape{12}{\magstephalf}{OT1TT}
\setfont\ssecttsl\ttslshape{10}{1315}{OT1TT}
\setfont\ssecsf\sfbshape{12}{\magstephalf}{OT1}
\let\ssecbf\ssecrm
\setfont\ssecsc\scbshape{10}{1315}{OT1}
\font\sseci=cmmi12 scaled \magstephalf
\font\ssecsy=cmsy10 scaled 1315
\def\ssececsize{1200}

% Reduced fonts for @acro in text (10pt).
\def\reducednominalsize{10pt}
\setfont\reducedrm\rmshape{10}{1000}{OT1}
\setfont\reducedtt\ttshape{10}{1000}{OT1TT}
\setfont\reducedbf\bfshape{10}{1000}{OT1}
\setfont\reducedit\itshape{10}{1000}{OT1IT}
\setfont\reducedsl\slshape{10}{1000}{OT1}
\setfont\reducedsf\sfshape{10}{1000}{OT1}
\setfont\reducedsc\scshape{10}{1000}{OT1}
\setfont\reducedttsl\ttslshape{10}{1000}{OT1TT}
\font\reducedi=cmmi10
\font\reducedsy=cmsy10
\def\reducedecsize{1000}

\textleading = 13.2pt % line spacing for 11pt CM
\textfonts            % reset the current fonts
\rm
} % end of 11pt text font size definitions, \definetextfontsizexi


% Definitions to make the main text be 10pt Computer Modern, with
% section, chapter, etc., sizes following suit.  This is for the GNU
% Press printing of the Emacs 22 manual.  Maybe other manuals in the
% future.  Used with @smallbook, which sets the leading to 12pt.
%
\def\definetextfontsizex{%
% Text fonts (10pt).
\def\textnominalsize{10pt}
\edef\mainmagstep{1000}
\setfont\textrm\rmshape{10}{\mainmagstep}{OT1}
\setfont\texttt\ttshape{10}{\mainmagstep}{OT1TT}
\setfont\textbf\bfshape{10}{\mainmagstep}{OT1}
\setfont\textit\itshape{10}{\mainmagstep}{OT1IT}
\setfont\textsl\slshape{10}{\mainmagstep}{OT1}
\setfont\textsf\sfshape{10}{\mainmagstep}{OT1}
\setfont\textsc\scshape{10}{\mainmagstep}{OT1}
\setfont\textttsl\ttslshape{10}{\mainmagstep}{OT1TT}
\font\texti=cmmi10 scaled \mainmagstep
\font\textsy=cmsy10 scaled \mainmagstep
\def\textecsize{1000}

% A few fonts for @defun names and args.
\setfont\defbf\bfshape{10}{\magstephalf}{OT1}
\setfont\deftt\ttshape{10}{\magstephalf}{OT1TT}
\setfont\defttsl\ttslshape{10}{\magstephalf}{OT1TT}
\def\df{\let\tentt=\deftt \let\tenbf = \defbf \let\tenttsl=\defttsl \bf}

% Fonts for indices, footnotes, small examples (9pt).
\def\smallnominalsize{9pt}
\setfont\smallrm\rmshape{9}{1000}{OT1}
\setfont\smalltt\ttshape{9}{1000}{OT1TT}
\setfont\smallbf\bfshape{10}{900}{OT1}
\setfont\smallit\itshape{9}{1000}{OT1IT}
\setfont\smallsl\slshape{9}{1000}{OT1}
\setfont\smallsf\sfshape{9}{1000}{OT1}
\setfont\smallsc\scshape{10}{900}{OT1}
\setfont\smallttsl\ttslshape{10}{900}{OT1TT}
\font\smalli=cmmi9
\font\smallsy=cmsy9
\def\smallecsize{0900}

% Fonts for small examples (8pt).
\def\smallernominalsize{8pt}
\setfont\smallerrm\rmshape{8}{1000}{OT1}
\setfont\smallertt\ttshape{8}{1000}{OT1TT}
\setfont\smallerbf\bfshape{10}{800}{OT1}
\setfont\smallerit\itshape{8}{1000}{OT1IT}
\setfont\smallersl\slshape{8}{1000}{OT1}
\setfont\smallersf\sfshape{8}{1000}{OT1}
\setfont\smallersc\scshape{10}{800}{OT1}
\setfont\smallerttsl\ttslshape{10}{800}{OT1TT}
\font\smalleri=cmmi8
\font\smallersy=cmsy8
\def\smallerecsize{0800}

% Fonts for title page (20.4pt):
\def\titlenominalsize{20pt}
\setfont\titlerm\rmbshape{12}{\magstep3}{OT1}
\setfont\titleit\itbshape{10}{\magstep4}{OT1IT}
\setfont\titlesl\slbshape{10}{\magstep4}{OT1}
\setfont\titlett\ttbshape{12}{\magstep3}{OT1TT}
\setfont\titlettsl\ttslshape{10}{\magstep4}{OT1TT}
\setfont\titlesf\sfbshape{17}{\magstep1}{OT1}
\let\titlebf=\titlerm
\setfont\titlesc\scbshape{10}{\magstep4}{OT1}
\font\titlei=cmmi12 scaled \magstep3
\font\titlesy=cmsy10 scaled \magstep4
\def\titleecsize{2074}

% Chapter fonts (14.4pt).
\def\chapnominalsize{14pt}
\setfont\chaprm\rmbshape{12}{\magstep1}{OT1}
\setfont\chapit\itbshape{10}{\magstep2}{OT1IT}
\setfont\chapsl\slbshape{10}{\magstep2}{OT1}
\setfont\chaptt\ttbshape{12}{\magstep1}{OT1TT}
\setfont\chapttsl\ttslshape{10}{\magstep2}{OT1TT}
\setfont\chapsf\sfbshape{12}{\magstep1}{OT1}
\let\chapbf\chaprm
\setfont\chapsc\scbshape{10}{\magstep2}{OT1}
\font\chapi=cmmi12 scaled \magstep1
\font\chapsy=cmsy10 scaled \magstep2
\def\chapecsize{1440}

% Section fonts (12pt).
\def\secnominalsize{12pt}
\setfont\secrm\rmbshape{12}{1000}{OT1}
\setfont\secit\itbshape{10}{\magstep1}{OT1IT}
\setfont\secsl\slbshape{10}{\magstep1}{OT1}
\setfont\sectt\ttbshape{12}{1000}{OT1TT}
\setfont\secttsl\ttslshape{10}{\magstep1}{OT1TT}
\setfont\secsf\sfbshape{12}{1000}{OT1}
\let\secbf\secrm
\setfont\secsc\scbshape{10}{\magstep1}{OT1}
\font\seci=cmmi12
\font\secsy=cmsy10 scaled \magstep1
\def\sececsize{1200}

% Subsection fonts (10pt).
\def\ssecnominalsize{10pt}
\setfont\ssecrm\rmbshape{10}{1000}{OT1}
\setfont\ssecit\itbshape{10}{1000}{OT1IT}
\setfont\ssecsl\slbshape{10}{1000}{OT1}
\setfont\ssectt\ttbshape{10}{1000}{OT1TT}
\setfont\ssecttsl\ttslshape{10}{1000}{OT1TT}
\setfont\ssecsf\sfbshape{10}{1000}{OT1}
\let\ssecbf\ssecrm
\setfont\ssecsc\scbshape{10}{1000}{OT1}
\font\sseci=cmmi10
\font\ssecsy=cmsy10
\def\ssececsize{1000}

% Reduced fonts for @acro in text (9pt).
\def\reducednominalsize{9pt}
\setfont\reducedrm\rmshape{9}{1000}{OT1}
\setfont\reducedtt\ttshape{9}{1000}{OT1TT}
\setfont\reducedbf\bfshape{10}{900}{OT1}
\setfont\reducedit\itshape{9}{1000}{OT1IT}
\setfont\reducedsl\slshape{9}{1000}{OT1}
\setfont\reducedsf\sfshape{9}{1000}{OT1}
\setfont\reducedsc\scshape{10}{900}{OT1}
\setfont\reducedttsl\ttslshape{10}{900}{OT1TT}
\font\reducedi=cmmi9
\font\reducedsy=cmsy9
\def\reducedecsize{0900}

\divide\parskip by 2  % reduce space between paragraphs
\textleading = 12pt   % line spacing for 10pt CM
\textfonts            % reset the current fonts
\rm
} % end of 10pt text font size definitions, \definetextfontsizex


% We provide the user-level command
%   @fonttextsize 10
% (or 11) to redefine the text font size.  pt is assumed.
%
\def\xiword{11}
\def\xword{10}
\def\xwordpt{10pt}
%
\parseargdef\fonttextsize{%
  \def\textsizearg{#1}%
  %\wlog{doing @fonttextsize \textsizearg}%
  %
  % Set \globaldefs so that documents can use this inside @tex, since
  % makeinfo 4.8 does not support it, but we need it nonetheless.
  %
 \begingroup \globaldefs=1
  \ifx\textsizearg\xword \definetextfontsizex
  \else \ifx\textsizearg\xiword \definetextfontsizexi
  \else
    \errhelp=\EMsimple
    \errmessage{@fonttextsize only supports `10' or `11', not `\textsizearg'}
  \fi\fi
 \endgroup
}


% In order for the font changes to affect most math symbols and letters,
% we have to define the \textfont of the standard families.  Since
% texinfo doesn't allow for producing subscripts and superscripts except
% in the main text, we don't bother to reset \scriptfont and
% \scriptscriptfont (which would also require loading a lot more fonts).
%
\def\resetmathfonts{%
  \textfont0=\tenrm \textfont1=\teni \textfont2=\tensy
  \textfont\itfam=\tenit \textfont\slfam=\tensl \textfont\bffam=\tenbf
  \textfont\ttfam=\tentt \textfont\sffam=\tensf
}

% The font-changing commands redefine the meanings of \tenSTYLE, instead
% of just \STYLE.  We do this because \STYLE needs to also set the
% current \fam for math mode.  Our \STYLE (e.g., \rm) commands hardwire
% \tenSTYLE to set the current font.
%
% Each font-changing command also sets the names \lsize (one size lower)
% and \lllsize (three sizes lower).  These relative commands are used in
% the LaTeX logo and acronyms.
%
% This all needs generalizing, badly.
%
\def\textfonts{%
  \let\tenrm=\textrm \let\tenit=\textit \let\tensl=\textsl
  \let\tenbf=\textbf \let\tentt=\texttt \let\smallcaps=\textsc
  \let\tensf=\textsf \let\teni=\texti \let\tensy=\textsy
  \let\tenttsl=\textttsl
  \def\curfontsize{text}%
  \def\lsize{reduced}\def\lllsize{smaller}%
  \resetmathfonts \setleading{\textleading}}
\def\titlefonts{%
  \let\tenrm=\titlerm \let\tenit=\titleit \let\tensl=\titlesl
  \let\tenbf=\titlebf \let\tentt=\titlett \let\smallcaps=\titlesc
  \let\tensf=\titlesf \let\teni=\titlei \let\tensy=\titlesy
  \let\tenttsl=\titlettsl
  \def\curfontsize{title}%
  \def\lsize{chap}\def\lllsize{subsec}%
  \resetmathfonts \setleading{27pt}}
\def\titlefont#1{{\titlefonts\rmisbold #1}}
\def\chapfonts{%
  \let\tenrm=\chaprm \let\tenit=\chapit \let\tensl=\chapsl
  \let\tenbf=\chapbf \let\tentt=\chaptt \let\smallcaps=\chapsc
  \let\tensf=\chapsf \let\teni=\chapi \let\tensy=\chapsy
  \let\tenttsl=\chapttsl
  \def\curfontsize{chap}%
  \def\lsize{sec}\def\lllsize{text}%
  \resetmathfonts \setleading{19pt}}
\def\secfonts{%
  \let\tenrm=\secrm \let\tenit=\secit \let\tensl=\secsl
  \let\tenbf=\secbf \let\tentt=\sectt \let\smallcaps=\secsc
  \let\tensf=\secsf \let\teni=\seci \let\tensy=\secsy
  \let\tenttsl=\secttsl
  \def\curfontsize{sec}%
  \def\lsize{subsec}\def\lllsize{reduced}%
  \resetmathfonts \setleading{16pt}}
\def\subsecfonts{%
  \let\tenrm=\ssecrm \let\tenit=\ssecit \let\tensl=\ssecsl
  \let\tenbf=\ssecbf \let\tentt=\ssectt \let\smallcaps=\ssecsc
  \let\tensf=\ssecsf \let\teni=\sseci \let\tensy=\ssecsy
  \let\tenttsl=\ssecttsl
  \def\curfontsize{ssec}%
  \def\lsize{text}\def\lllsize{small}%
  \resetmathfonts \setleading{15pt}}
\let\subsubsecfonts = \subsecfonts
\def\reducedfonts{%
  \let\tenrm=\reducedrm \let\tenit=\reducedit \let\tensl=\reducedsl
  \let\tenbf=\reducedbf \let\tentt=\reducedtt \let\reducedcaps=\reducedsc
  \let\tensf=\reducedsf \let\teni=\reducedi \let\tensy=\reducedsy
  \let\tenttsl=\reducedttsl
  \def\curfontsize{reduced}%
  \def\lsize{small}\def\lllsize{smaller}%
  \resetmathfonts \setleading{10.5pt}}
\def\smallfonts{%
  \let\tenrm=\smallrm \let\tenit=\smallit \let\tensl=\smallsl
  \let\tenbf=\smallbf \let\tentt=\smalltt \let\smallcaps=\smallsc
  \let\tensf=\smallsf \let\teni=\smalli \let\tensy=\smallsy
  \let\tenttsl=\smallttsl
  \def\curfontsize{small}%
  \def\lsize{smaller}\def\lllsize{smaller}%
  \resetmathfonts \setleading{10.5pt}}
\def\smallerfonts{%
  \let\tenrm=\smallerrm \let\tenit=\smallerit \let\tensl=\smallersl
  \let\tenbf=\smallerbf \let\tentt=\smallertt \let\smallcaps=\smallersc
  \let\tensf=\smallersf \let\teni=\smalleri \let\tensy=\smallersy
  \let\tenttsl=\smallerttsl
  \def\curfontsize{smaller}%
  \def\lsize{smaller}\def\lllsize{smaller}%
  \resetmathfonts \setleading{9.5pt}}

% Fonts for short table of contents.
\setfont\shortcontrm\rmshape{12}{1000}{OT1}
\setfont\shortcontbf\bfshape{10}{\magstep1}{OT1}  % no cmb12
\setfont\shortcontsl\slshape{12}{1000}{OT1}
\setfont\shortconttt\ttshape{12}{1000}{OT1TT}

% Define these just so they can be easily changed for other fonts.
\def\angleleft{$\langle$}
\def\angleright{$\rangle$}

% Set the fonts to use with the @small... environments.
\let\smallexamplefonts = \smallfonts

% About \smallexamplefonts.  If we use \smallfonts (9pt), @smallexample
% can fit this many characters:
%   8.5x11=86   smallbook=72  a4=90  a5=69
% If we use \scriptfonts (8pt), then we can fit this many characters:
%   8.5x11=90+  smallbook=80  a4=90+  a5=77
% For me, subjectively, the few extra characters that fit aren't worth
% the additional smallness of 8pt.  So I'm making the default 9pt.
%
% By the way, for comparison, here's what fits with @example (10pt):
%   8.5x11=71  smallbook=60  a4=75  a5=58
% --karl, 24jan03.

% Set up the default fonts, so we can use them for creating boxes.
%
\definetextfontsizexi


\message{markup,}

% Check if we are currently using a typewriter font.  Since all the
% Computer Modern typewriter fonts have zero interword stretch (and
% shrink), and it is reasonable to expect all typewriter fonts to have
% this property, we can check that font parameter.
%
\def\ifmonospace{\ifdim\fontdimen3\font=0pt }

% Markup style infrastructure.  \defmarkupstylesetup\INITMACRO will
% define and register \INITMACRO to be called on markup style changes.
% \INITMACRO can check \currentmarkupstyle for the innermost
% style and the set of \ifmarkupSTYLE switches for all styles
% currently in effect.
\newif\ifmarkupvar
\newif\ifmarkupsamp
\newif\ifmarkupkey
%\newif\ifmarkupfile % @file == @samp.
%\newif\ifmarkupoption % @option == @samp.
\newif\ifmarkupcode
\newif\ifmarkupkbd
%\newif\ifmarkupenv % @env == @code.
%\newif\ifmarkupcommand % @command == @code.
\newif\ifmarkuptex % @tex (and part of @math, for now).
\newif\ifmarkupexample
\newif\ifmarkupverb
\newif\ifmarkupverbatim

\let\currentmarkupstyle\empty

\def\setupmarkupstyle#1{%
  \csname markup#1true\endcsname
  \def\currentmarkupstyle{#1}%
  \markupstylesetup
}

\let\markupstylesetup\empty

\def\defmarkupstylesetup#1{%
  \expandafter\def\expandafter\markupstylesetup
    \expandafter{\markupstylesetup #1}%
  \def#1%
}

% Markup style setup for left and right quotes.
\defmarkupstylesetup\markupsetuplq{%
  \expandafter\let\expandafter \temp
    \csname markupsetuplq\currentmarkupstyle\endcsname
  \ifx\temp\relax \markupsetuplqdefault \else \temp \fi
}

\defmarkupstylesetup\markupsetuprq{%
  \expandafter\let\expandafter \temp
    \csname markupsetuprq\currentmarkupstyle\endcsname
  \ifx\temp\relax \markupsetuprqdefault \else \temp \fi
}

{
\catcode`\'=\active
\catcode`\`=\active

\gdef\markupsetuplqdefault{\let`\lq}
\gdef\markupsetuprqdefault{\let'\rq}

\gdef\markupsetcodequoteleft{\let`\codequoteleft}
\gdef\markupsetcodequoteright{\let'\codequoteright}
}

\let\markupsetuplqcode \markupsetcodequoteleft
\let\markupsetuprqcode \markupsetcodequoteright
%
\let\markupsetuplqexample \markupsetcodequoteleft
\let\markupsetuprqexample \markupsetcodequoteright
%
\let\markupsetuplqkbd     \markupsetcodequoteleft
\let\markupsetuprqkbd     \markupsetcodequoteright
%
\let\markupsetuplqsamp \markupsetcodequoteleft
\let\markupsetuprqsamp \markupsetcodequoteright
%
\let\markupsetuplqverb \markupsetcodequoteleft
\let\markupsetuprqverb \markupsetcodequoteright
%
\let\markupsetuplqverbatim \markupsetcodequoteleft
\let\markupsetuprqverbatim \markupsetcodequoteright

% Allow an option to not use regular directed right quote/apostrophe
% (char 0x27), but instead the undirected quote from cmtt (char 0x0d).
% The undirected quote is ugly, so don't make it the default, but it
% works for pasting with more pdf viewers (at least evince), the
% lilypond developers report.  xpdf does work with the regular 0x27.
%
\def\codequoteright{%
  \expandafter\ifx\csname SETtxicodequoteundirected\endcsname\relax
    \expandafter\ifx\csname SETcodequoteundirected\endcsname\relax
      '%
    \else \char'15 \fi
  \else \char'15 \fi
}
%
% and a similar option for the left quote char vs. a grave accent.
% Modern fonts display ASCII 0x60 as a grave accent, so some people like
% the code environments to do likewise.
%
\def\codequoteleft{%
  \expandafter\ifx\csname SETtxicodequotebacktick\endcsname\relax
    \expandafter\ifx\csname SETcodequotebacktick\endcsname\relax
      % [Knuth] pp. 380,381,391
      % \relax disables Spanish ligatures ?` and !` of \tt font.
      \relax`%
    \else \char'22 \fi
  \else \char'22 \fi
}

% Commands to set the quote options.
% 
\parseargdef\codequoteundirected{%
  \def\temp{#1}%
  \ifx\temp\onword
    \expandafter\let\csname SETtxicodequoteundirected\endcsname
      = t%
  \else\ifx\temp\offword
    \expandafter\let\csname SETtxicodequoteundirected\endcsname
      = \relax
  \else
    \errhelp = \EMsimple
    \errmessage{Unknown @codequoteundirected value `\temp', must be on|off}%
  \fi\fi
}
%
\parseargdef\codequotebacktick{%
  \def\temp{#1}%
  \ifx\temp\onword
    \expandafter\let\csname SETtxicodequotebacktick\endcsname
      = t%
  \else\ifx\temp\offword
    \expandafter\let\csname SETtxicodequotebacktick\endcsname
      = \relax
  \else
    \errhelp = \EMsimple
    \errmessage{Unknown @codequotebacktick value `\temp', must be on|off}%
  \fi\fi
}

% [Knuth] pp. 380,381,391, disable Spanish ligatures ?` and !` of \tt font.
\def\noligaturesquoteleft{\relax\lq}

% Count depth in font-changes, for error checks
\newcount\fontdepth \fontdepth=0

% Font commands.

% #1 is the font command (\sl or \it), #2 is the text to slant.
% If we are in a monospaced environment, however, 1) always use \ttsl,
% and 2) do not add an italic correction.
\def\dosmartslant#1#2{%
  \ifusingtt 
    {{\ttsl #2}\let\next=\relax}%
    {\def\next{{#1#2}\futurelet\next\smartitaliccorrection}}%
  \next
}
\def\smartslanted{\dosmartslant\sl}
\def\smartitalic{\dosmartslant\it}

% Output an italic correction unless \next (presumed to be the following
% character) is such as not to need one.
\def\smartitaliccorrection{%
  \ifx\next,%
  \else\ifx\next-%
  \else\ifx\next.%
  \else\ptexslash
  \fi\fi\fi
  \aftersmartic
}

% Unconditional use \ttsl, and no ic.  @var is set to this for defuns.
\def\ttslanted#1{{\ttsl #1}}

% @cite is like \smartslanted except unconditionally use \sl.  We never want
% ttsl for book titles, do we?
\def\cite#1{{\sl #1}\futurelet\next\smartitaliccorrection}

\def\aftersmartic{}
\def\var#1{%
  \let\saveaftersmartic = \aftersmartic
  \def\aftersmartic{\null\let\aftersmartic=\saveaftersmartic}%
  \smartslanted{#1}%
}

\let\i=\smartitalic
\let\slanted=\smartslanted
\let\dfn=\smartslanted
\let\emph=\smartitalic

% Explicit font changes: @r, @sc, undocumented @ii.
\def\r#1{{\rm #1}}              % roman font
\def\sc#1{{\smallcaps#1}}       % smallcaps font
\def\ii#1{{\it #1}}             % italic font

% @b, explicit bold.  Also @strong.
\def\b#1{{\bf #1}}
\let\strong=\b

% @sansserif, explicit sans.
\def\sansserif#1{{\sf #1}}

% We can't just use \exhyphenpenalty, because that only has effect at
% the end of a paragraph.  Restore normal hyphenation at the end of the
% group within which \nohyphenation is presumably called.
%
\def\nohyphenation{\hyphenchar\font = -1  \aftergroup\restorehyphenation}
\def\restorehyphenation{\hyphenchar\font = `- }

% Set sfcode to normal for the chars that usually have another value.
% Can't use plain's \frenchspacing because it uses the `\x notation, and
% sometimes \x has an active definition that messes things up.
%
\catcode`@=11
  \def\plainfrenchspacing{%
    \sfcode\dotChar  =\@m \sfcode\questChar=\@m \sfcode\exclamChar=\@m
    \sfcode\colonChar=\@m \sfcode\semiChar =\@m \sfcode\commaChar =\@m
    \def\endofsentencespacefactor{1000}% for @. and friends
  }
  \def\plainnonfrenchspacing{%
    \sfcode`\.3000\sfcode`\?3000\sfcode`\!3000
    \sfcode`\:2000\sfcode`\;1500\sfcode`\,1250
    \def\endofsentencespacefactor{3000}% for @. and friends
  }
\catcode`@=\other
\def\endofsentencespacefactor{3000}% default

% @t, explicit typewriter.
\def\t#1{%
  {\tt \rawbackslash \plainfrenchspacing #1}%
  \null
}

% @samp.
\def\samp#1{{\setupmarkupstyle{samp}\lq\tclose{#1}\rq\null}}

% @indicateurl is \samp, that is, with quotes.
\let\indicateurl=\samp

% @code (and similar) prints in typewriter, but with spaces the same
% size as normal in the surrounding text, without hyphenation, etc.
% This is a subroutine for that.
\def\tclose#1{%
  {%
    % Change normal interword space to be same as for the current font.
    \spaceskip = \fontdimen2\font
    %
    % Switch to typewriter.
    \tt
    %
    % But `\ ' produces the large typewriter interword space.
    \def\ {{\spaceskip = 0pt{} }}%
    %
    % Turn off hyphenation.
    \nohyphenation
    %
    \rawbackslash
    \plainfrenchspacing
    #1%
  }%
  \null % reset spacefactor to 1000
}

% We *must* turn on hyphenation at `-' and `_' in @code.
% Otherwise, it is too hard to avoid overfull hboxes
% in the Emacs manual, the Library manual, etc.
%
% Unfortunately, TeX uses one parameter (\hyphenchar) to control
% both hyphenation at - and hyphenation within words.
% We must therefore turn them both off (\tclose does that)
% and arrange explicitly to hyphenate at a dash.
%  -- rms.
{
  \catcode`\-=\active \catcode`\_=\active
  \catcode`\'=\active \catcode`\`=\active
  \global\let'=\rq \global\let`=\lq  % default definitions
  %
  \global\def\code{\begingroup
    \setupmarkupstyle{code}%
    % The following should really be moved into \setupmarkupstyle handlers.
    \catcode\dashChar=\active  \catcode\underChar=\active
    \ifallowcodebreaks
     \let-\codedash
     \let_\codeunder
    \else
     \let-\normaldash
     \let_\realunder
    \fi
    \codex
  }
}

\def\codex #1{\tclose{#1}\endgroup}

\def\normaldash{-}
\def\codedash{-\discretionary{}{}{}}
\def\codeunder{%
  % this is all so @math{@code{var_name}+1} can work.  In math mode, _
  % is "active" (mathcode"8000) and \normalunderscore (or \char95, etc.)
  % will therefore expand the active definition of _, which is us
  % (inside @code that is), therefore an endless loop.
  \ifusingtt{\ifmmode
               \mathchar"075F % class 0=ordinary, family 7=ttfam, pos 0x5F=_.
             \else\normalunderscore \fi
             \discretionary{}{}{}}%
            {\_}%
}

% An additional complication: the above will allow breaks after, e.g.,
% each of the four underscores in __typeof__.  This is bad.
% @allowcodebreaks provides a document-level way to turn breaking at -
% and _ on and off.
%
\newif\ifallowcodebreaks  \allowcodebreakstrue

\def\keywordtrue{true}
\def\keywordfalse{false}

\parseargdef\allowcodebreaks{%
  \def\txiarg{#1}%
  \ifx\txiarg\keywordtrue
    \allowcodebreakstrue
  \else\ifx\txiarg\keywordfalse
    \allowcodebreaksfalse
  \else
    \errhelp = \EMsimple
    \errmessage{Unknown @allowcodebreaks option `\txiarg', must be true|false}%
  \fi\fi
}

% For @command, @env, @file, @option quotes seem unnecessary,
% so use \code rather than \samp.
\let\command=\code
\let\env=\code
\let\file=\code
\let\option=\code

% @uref (abbreviation for `urlref') takes an optional (comma-separated)
% second argument specifying the text to display and an optional third
% arg as text to display instead of (rather than in addition to) the url
% itself.  First (mandatory) arg is the url.
% (This \urefnobreak definition isn't used now, leaving it for a while
% for comparison.)
\def\urefnobreak#1{\dourefnobreak #1,,,\finish}
\def\dourefnobreak#1,#2,#3,#4\finish{\begingroup
  \unsepspaces
  \pdfurl{#1}%
  \setbox0 = \hbox{\ignorespaces #3}%
  \ifdim\wd0 > 0pt
    \unhbox0 % third arg given, show only that
  \else
    \setbox0 = \hbox{\ignorespaces #2}%
    \ifdim\wd0 > 0pt
      \ifpdf
        \unhbox0             % PDF: 2nd arg given, show only it
      \else
        \unhbox0\ (\code{#1})% DVI: 2nd arg given, show both it and url
      \fi
    \else
      \code{#1}% only url given, so show it
    \fi
  \fi
  \endlink
\endgroup}

% This \urefbreak definition is the active one.
\def\urefbreak{\begingroup \urefcatcodes \dourefbreak}
\let\uref=\urefbreak
\def\dourefbreak#1{\urefbreakfinish #1,,,\finish}
\def\urefbreakfinish#1,#2,#3,#4\finish{% doesn't work in @example
  \unsepspaces
  \pdfurl{#1}%
  \setbox0 = \hbox{\ignorespaces #3}%
  \ifdim\wd0 > 0pt
    \unhbox0 % third arg given, show only that
  \else
    \setbox0 = \hbox{\ignorespaces #2}%
    \ifdim\wd0 > 0pt
      \ifpdf
        \unhbox0             % PDF: 2nd arg given, show only it
      \else
        \unhbox0\ (\urefcode{#1})% DVI: 2nd arg given, show both it and url
      \fi
    \else
      \urefcode{#1}% only url given, so show it
    \fi
  \fi
  \endlink
\endgroup}

% Allow line breaks around only a few characters (only).
\def\urefcatcodes{%
  \catcode\ampChar=\active   \catcode\dotChar=\active
  \catcode\hashChar=\active  \catcode\questChar=\active
  \catcode\slashChar=\active
}
{
  \urefcatcodes
  %
  \global\def\urefcode{\begingroup
    \setupmarkupstyle{code}%
    \urefcatcodes
    \let&\urefcodeamp
    \let.\urefcodedot
    \let#\urefcodehash
    \let?\urefcodequest
    \let/\urefcodeslash
    \codex
  }
  %
  % By default, they are just regular characters.
  \global\def&{\normalamp}
  \global\def.{\normaldot}
  \global\def#{\normalhash}
  \global\def?{\normalquest}
  \global\def/{\normalslash}
}

% we put a little stretch before and after the breakable chars, to help
% line breaking of long url's.  The unequal skips make look better in
% cmtt at least, especially for dots.
\def\urefprestretch{\urefprebreak \hskip0pt plus.13em }
\def\urefpoststretch{\urefpostbreak \hskip0pt plus.1em }
%
\def\urefcodeamp{\urefprestretch \&\urefpoststretch}
\def\urefcodedot{\urefprestretch .\urefpoststretch}
\def\urefcodehash{\urefprestretch \#\urefpoststretch}
\def\urefcodequest{\urefprestretch ?\urefpoststretch}
\def\urefcodeslash{\futurelet\next\urefcodeslashfinish}
{
  \catcode`\/=\active
  \global\def\urefcodeslashfinish{%
    \urefprestretch \slashChar
    % Allow line break only after the final / in a sequence of
    % slashes, to avoid line break between the slashes in http://.
    \ifx\next/\else \urefpoststretch \fi
  }
}

% One more complication: by default we'll break after the special
% characters, but some people like to break before the special chars, so
% allow that.  Also allow no breaking at all, for manual control.
% 
\parseargdef\urefbreakstyle{%
  \def\txiarg{#1}%
  \ifx\txiarg\wordnone
    \def\urefprebreak{\nobreak}\def\urefpostbreak{\nobreak}
  \else\ifx\txiarg\wordbefore
    \def\urefprebreak{\allowbreak}\def\urefpostbreak{\nobreak}
  \else\ifx\txiarg\wordafter
    \def\urefprebreak{\nobreak}\def\urefpostbreak{\allowbreak}
  \else
    \errhelp = \EMsimple
    \errmessage{Unknown @urefbreakstyle setting `\txiarg'}%
  \fi\fi\fi
}
\def\wordafter{after}
\def\wordbefore{before}
\def\wordnone{none}

\urefbreakstyle after

% @url synonym for @uref, since that's how everyone uses it.
%
\let\url=\uref

% rms does not like angle brackets --karl, 17may97.
% So now @email is just like @uref, unless we are pdf.
%
%\def\email#1{\angleleft{\tt #1}\angleright}
\ifpdf
  \def\email#1{\doemail#1,,\finish}
  \def\doemail#1,#2,#3\finish{\begingroup
    \unsepspaces
    \pdfurl{mailto:#1}%
    \setbox0 = \hbox{\ignorespaces #2}%
    \ifdim\wd0>0pt\unhbox0\else\code{#1}\fi
    \endlink
  \endgroup}
\else
  \let\email=\uref
\fi

% @kbdinputstyle -- arg is `distinct' (@kbd uses slanted tty font always),
%   `example' (@kbd uses ttsl only inside of @example and friends),
%   or `code' (@kbd uses normal tty font always).
\parseargdef\kbdinputstyle{%
  \def\txiarg{#1}%
  \ifx\txiarg\worddistinct
    \gdef\kbdexamplefont{\ttsl}\gdef\kbdfont{\ttsl}%
  \else\ifx\txiarg\wordexample
    \gdef\kbdexamplefont{\ttsl}\gdef\kbdfont{\tt}%
  \else\ifx\txiarg\wordcode
    \gdef\kbdexamplefont{\tt}\gdef\kbdfont{\tt}%
  \else
    \errhelp = \EMsimple
    \errmessage{Unknown @kbdinputstyle setting `\txiarg'}%
  \fi\fi\fi
}
\def\worddistinct{distinct}
\def\wordexample{example}
\def\wordcode{code}

% Default is `distinct'.
\kbdinputstyle distinct

% @kbd is like @code, except that if the argument is just one @key command,
% then @kbd has no effect.
\def\kbd#1{{\def\look{#1}\expandafter\kbdsub\look??\par}}

\def\xkey{\key}
\def\kbdsub#1#2#3\par{%
  \def\one{#1}\def\three{#3}\def\threex{??}%
  \ifx\one\xkey\ifx\threex\three \key{#2}%
  \else{\tclose{\kbdfont\setupmarkupstyle{kbd}\look}}\fi
  \else{\tclose{\kbdfont\setupmarkupstyle{kbd}\look}}\fi
}

% definition of @key that produces a lozenge.  Doesn't adjust to text size.
%\setfont\keyrm\rmshape{8}{1000}{OT1}
%\font\keysy=cmsy9
%\def\key#1{{\keyrm\textfont2=\keysy \leavevmode\hbox{%
%  \raise0.4pt\hbox{\angleleft}\kern-.08em\vtop{%
%    \vbox{\hrule\kern-0.4pt
%     \hbox{\raise0.4pt\hbox{\vphantom{\angleleft}}#1}}%
%    \kern-0.4pt\hrule}%
%  \kern-.06em\raise0.4pt\hbox{\angleright}}}}

% definition of @key with no lozenge.  If the current font is already
% monospace, don't change it; that way, we respect @kbdinputstyle.  But
% if it isn't monospace, then use \tt.
%
\def\key#1{{\setupmarkupstyle{key}%
  \nohyphenation
  \ifmonospace\else\tt\fi
  #1}\null}

% @clicksequence{File @click{} Open ...}
\def\clicksequence#1{\begingroup #1\endgroup}

% @clickstyle @arrow   (by default)
\parseargdef\clickstyle{\def\click{#1}}
\def\click{\arrow}

% Typeset a dimension, e.g., `in' or `pt'.  The only reason for the
% argument is to make the input look right: @dmn{pt} instead of @dmn{}pt.
%
\def\dmn#1{\thinspace #1}

% @l was never documented to mean ``switch to the Lisp font'',
% and it is not used as such in any manual I can find.  We need it for
% Polish suppressed-l.  --karl, 22sep96.
%\def\l#1{{\li #1}\null}

% @acronym for "FBI", "NATO", and the like.
% We print this one point size smaller, since it's intended for
% all-uppercase.
%
\def\acronym#1{\doacronym #1,,\finish}
\def\doacronym#1,#2,#3\finish{%
  {\selectfonts\lsize #1}%
  \def\temp{#2}%
  \ifx\temp\empty \else
    \space ({\unsepspaces \ignorespaces \temp \unskip})%
  \fi
  \null % reset \spacefactor=1000
}

% @abbr for "Comput. J." and the like.
% No font change, but don't do end-of-sentence spacing.
%
\def\abbr#1{\doabbr #1,,\finish}
\def\doabbr#1,#2,#3\finish{%
  {\plainfrenchspacing #1}%
  \def\temp{#2}%
  \ifx\temp\empty \else
    \space ({\unsepspaces \ignorespaces \temp \unskip})%
  \fi
  \null % reset \spacefactor=1000
}

% @asis just yields its argument.  Used with @table, for example.
%
\def\asis#1{#1}

% @math outputs its argument in math mode.
%
% One complication: _ usually means subscripts, but it could also mean
% an actual _ character, as in @math{@var{some_variable} + 1}.  So make
% _ active, and distinguish by seeing if the current family is \slfam,
% which is what @var uses.
{
  \catcode`\_ = \active
  \gdef\mathunderscore{%
    \catcode`\_=\active
    \def_{\ifnum\fam=\slfam \_\else\sb\fi}%
  }
}
% Another complication: we want \\ (and @\) to output a math (or tt) \.
% FYI, plain.tex uses \\ as a temporary control sequence (for no
% particular reason), but this is not advertised and we don't care.
%
% The \mathchar is class=0=ordinary, family=7=ttfam, position=5C=\.
\def\mathbackslash{\ifnum\fam=\ttfam \mathchar"075C \else\backslash \fi}
%
\def\math{%
  \tex
  \mathunderscore
  \let\\ = \mathbackslash
  \mathactive
  % make the texinfo accent commands work in math mode
  \let\"=\ddot
  \let\'=\acute
  \let\==\bar
  \let\^=\hat
  \let\`=\grave
  \let\u=\breve
  \let\v=\check
  \let\~=\tilde
  \let\dotaccent=\dot
  $\finishmath
}
\def\finishmath#1{#1$\endgroup}  % Close the group opened by \tex.

% Some active characters (such as <) are spaced differently in math.
% We have to reset their definitions in case the @math was an argument
% to a command which sets the catcodes (such as @item or @section).
%
{
  \catcode`^ = \active
  \catcode`< = \active
  \catcode`> = \active
  \catcode`+ = \active
  \catcode`' = \active
  \gdef\mathactive{%
    \let^ = \ptexhat
    \let< = \ptexless
    \let> = \ptexgtr
    \let+ = \ptexplus
    \let' = \ptexquoteright
  }
}

% ctrl is no longer a Texinfo command, but leave this definition for fun.
\def\ctrl #1{{\tt \rawbackslash \hat}#1}

% @inlinefmt{FMTNAME,PROCESSED-TEXT} and @inlineraw{FMTNAME,RAW-TEXT}.
% Ignore unless FMTNAME == tex; then it is like @iftex and @tex,
% except specified as a normal braced arg, so no newlines to worry about.
% 
\def\outfmtnametex{tex}
%
\long\def\inlinefmt#1{\doinlinefmt #1,\finish}
\long\def\doinlinefmt#1,#2,\finish{%
  \def\inlinefmtname{#1}%
  \ifx\inlinefmtname\outfmtnametex \ignorespaces #2\fi
}
% For raw, must switch into @tex before parsing the argument, to avoid
% setting catcodes prematurely.  Doing it this way means that, for
% example, @inlineraw{html, foo{bar} gets a parse error instead of being
% ignored.  But this isn't important because if people want a literal
% *right* brace they would have to use a command anyway, so they may as
% well use a command to get a left brace too.  We could re-use the
% delimiter character idea from \verb, but it seems like overkill.
% 
\long\def\inlineraw{\tex \doinlineraw}
\long\def\doinlineraw#1{\doinlinerawtwo #1,\finish}
\def\doinlinerawtwo#1,#2,\finish{%
  \def\inlinerawname{#1}%
  \ifx\inlinerawname\outfmtnametex \ignorespaces #2\fi
  \endgroup % close group opened by \tex.
}


\message{glyphs,}
% and logos.

% @@ prints an @, as does @atchar{}.
\def\@{\char64 }
\let\atchar=\@

% @{ @} @lbracechar{} @rbracechar{} all generate brace characters.
% Unless we're in typewriter, use \ecfont because the CM text fonts do
% not have braces, and we don't want to switch into math.
\def\mylbrace{{\ifmonospace\else\ecfont\fi \char123}}
\def\myrbrace{{\ifmonospace\else\ecfont\fi \char125}}
\let\{=\mylbrace \let\lbracechar=\{
\let\}=\myrbrace \let\rbracechar=\}
\begingroup
  % Definitions to produce \{ and \} commands for indices,
  % and @{ and @} for the aux/toc files.
  \catcode`\{ = \other \catcode`\} = \other
  \catcode`\[ = 1 \catcode`\] = 2
  \catcode`\! = 0 \catcode`\\ = \other
  !gdef!lbracecmd[\{]%
  !gdef!rbracecmd[\}]%
  !gdef!lbraceatcmd[@{]%
  !gdef!rbraceatcmd[@}]%
!endgroup

% @comma{} to avoid , parsing problems.
\let\comma = ,

% Accents: @, @dotaccent @ringaccent @ubaraccent @udotaccent
% Others are defined by plain TeX: @` @' @" @^ @~ @= @u @v @H.
\let\, = \ptexc
\let\dotaccent = \ptexdot
\def\ringaccent#1{{\accent23 #1}}
\let\tieaccent = \ptext
\let\ubaraccent = \ptexb
\let\udotaccent = \d

% Other special characters: @questiondown @exclamdown @ordf @ordm
% Plain TeX defines: @AA @AE @O @OE @L (plus lowercase versions) @ss.
\def\questiondown{?`}
\def\exclamdown{!`}
\def\ordf{\leavevmode\raise1ex\hbox{\selectfonts\lllsize \underbar{a}}}
\def\ordm{\leavevmode\raise1ex\hbox{\selectfonts\lllsize \underbar{o}}}

% Dotless i and dotless j, used for accents.
\def\imacro{i}
\def\jmacro{j}
\def\dotless#1{%
  \def\temp{#1}%
  \ifx\temp\imacro \ifmmode\imath \else\ptexi \fi
  \else\ifx\temp\jmacro \ifmmode\jmath \else\j \fi
  \else \errmessage{@dotless can be used only with i or j}%
  \fi\fi
}

% The \TeX{} logo, as in plain, but resetting the spacing so that a
% period following counts as ending a sentence.  (Idea found in latex.)
%
\edef\TeX{\TeX \spacefactor=1000 }

% @LaTeX{} logo.  Not quite the same results as the definition in
% latex.ltx, since we use a different font for the raised A; it's most
% convenient for us to use an explicitly smaller font, rather than using
% the \scriptstyle font (since we don't reset \scriptstyle and
% \scriptscriptstyle).
%
\def\LaTeX{%
  L\kern-.36em
  {\setbox0=\hbox{T}%
   \vbox to \ht0{\hbox{%
     \ifx\textnominalsize\xwordpt
       % for 10pt running text, \lllsize (8pt) is too small for the A in LaTeX.
       % Revert to plain's \scriptsize, which is 7pt.
       \count255=\the\fam $\fam\count255 \scriptstyle A$%
     \else
       % For 11pt, we can use our lllsize.
       \selectfonts\lllsize A%
     \fi
     }%
     \vss
  }}%
  \kern-.15em
  \TeX
}

% Some math mode symbols.
\def\bullet{$\ptexbullet$}
\def\geq{\ifmmode \ge\else $\ge$\fi}
\def\leq{\ifmmode \le\else $\le$\fi}
\def\minus{\ifmmode -\else $-$\fi}

% @dots{} outputs an ellipsis using the current font.
% We do .5em per period so that it has the same spacing in the cm
% typewriter fonts as three actual period characters; on the other hand,
% in other typewriter fonts three periods are wider than 1.5em.  So do
% whichever is larger.
%
\def\dots{%
  \leavevmode
  \setbox0=\hbox{...}% get width of three periods
  \ifdim\wd0 > 1.5em
    \dimen0 = \wd0
  \else
    \dimen0 = 1.5em
  \fi
  \hbox to \dimen0{%
    \hskip 0pt plus.25fil
    .\hskip 0pt plus1fil
    .\hskip 0pt plus1fil
    .\hskip 0pt plus.5fil
  }%
}

% @enddots{} is an end-of-sentence ellipsis.
%
\def\enddots{%
  \dots
  \spacefactor=\endofsentencespacefactor
}

% @point{}, @result{}, @expansion{}, @print{}, @equiv{}.
%
% Since these characters are used in examples, they should be an even number of
% \tt widths. Each \tt character is 1en, so two makes it 1em.
%
\def\point{$\star$}
\def\arrow{\leavevmode\raise.05ex\hbox to 1em{\hfil$\rightarrow$\hfil}}
\def\result{\leavevmode\raise.05ex\hbox to 1em{\hfil$\Rightarrow$\hfil}}
\def\expansion{\leavevmode\hbox to 1em{\hfil$\mapsto$\hfil}}
\def\print{\leavevmode\lower.1ex\hbox to 1em{\hfil$\dashv$\hfil}}
\def\equiv{\leavevmode\hbox to 1em{\hfil$\ptexequiv$\hfil}}

% The @error{} command.
% Adapted from the TeXbook's \boxit.
%
\newbox\errorbox
%
{\tentt \global\dimen0 = 3em}% Width of the box.
\dimen2 = .55pt % Thickness of rules
% The text. (`r' is open on the right, `e' somewhat less so on the left.)
\setbox0 = \hbox{\kern-.75pt \reducedsf \putworderror\kern-1.5pt}
%
\setbox\errorbox=\hbox to \dimen0{\hfil
   \hsize = \dimen0 \advance\hsize by -5.8pt % Space to left+right.
   \advance\hsize by -2\dimen2 % Rules.
   \vbox{%
      \hrule height\dimen2
      \hbox{\vrule width\dimen2 \kern3pt          % Space to left of text.
         \vtop{\kern2.4pt \box0 \kern2.4pt}% Space above/below.
         \kern3pt\vrule width\dimen2}% Space to right.
      \hrule height\dimen2}
    \hfil}
%
\def\error{\leavevmode\lower.7ex\copy\errorbox}

% @pounds{} is a sterling sign, which Knuth put in the CM italic font.
%
\def\pounds{{\it\$}}

% @euro{} comes from a separate font, depending on the current style.
% We use the free feym* fonts from the eurosym package by Henrik
% Theiling, which support regular, slanted, bold and bold slanted (and
% "outlined" (blackboard board, sort of) versions, which we don't need).
% It is available from http://www.ctan.org/tex-archive/fonts/eurosym.
%
% Although only regular is the truly official Euro symbol, we ignore
% that.  The Euro is designed to be slightly taller than the regular
% font height.
%
% feymr - regular
% feymo - slanted
% feybr - bold
% feybo - bold slanted
%
% There is no good (free) typewriter version, to my knowledge.
% A feymr10 euro is ~7.3pt wide, while a normal cmtt10 char is ~5.25pt wide.
% Hmm.
%
% Also doesn't work in math.  Do we need to do math with euro symbols?
% Hope not.
%
%
\def\euro{{\eurofont e}}
\def\eurofont{%
  % We set the font at each command, rather than predefining it in
  % \textfonts and the other font-switching commands, so that
  % installations which never need the symbol don't have to have the
  % font installed.
  %
  % There is only one designed size (nominal 10pt), so we always scale
  % that to the current nominal size.
  %
  % By the way, simply using "at 1em" works for cmr10 and the like, but
  % does not work for cmbx10 and other extended/shrunken fonts.
  %
  \def\eurosize{\csname\curfontsize nominalsize\endcsname}%
  %
  \ifx\curfontstyle\bfstylename
    % bold:
    \font\thiseurofont = \ifusingit{feybo10}{feybr10} at \eurosize
  \else
    % regular:
    \font\thiseurofont = \ifusingit{feymo10}{feymr10} at \eurosize
  \fi
  \thiseurofont
}

% Glyphs from the EC fonts.  We don't use \let for the aliases, because
% sometimes we redefine the original macro, and the alias should reflect
% the redefinition.
%
% Use LaTeX names for the Icelandic letters.
\def\DH{{\ecfont \char"D0}} % Eth
\def\dh{{\ecfont \char"F0}} % eth
\def\TH{{\ecfont \char"DE}} % Thorn
\def\th{{\ecfont \char"FE}} % thorn
%
\def\guillemetleft{{\ecfont \char"13}}
\def\guillemotleft{\guillemetleft}
\def\guillemetright{{\ecfont \char"14}}
\def\guillemotright{\guillemetright}
\def\guilsinglleft{{\ecfont \char"0E}}
\def\guilsinglright{{\ecfont \char"0F}}
\def\quotedblbase{{\ecfont \char"12}}
\def\quotesinglbase{{\ecfont \char"0D}}
%
% This positioning is not perfect (see the ogonek LaTeX package), but
% we have the precomposed glyphs for the most common cases.  We put the
% tests to use those glyphs in the single \ogonek macro so we have fewer
% dummy definitions to worry about for index entries, etc.
%
% ogonek is also used with other letters in Lithuanian (IOU), but using
% the precomposed glyphs for those is not so easy since they aren't in
% the same EC font.
\def\ogonek#1{{%
  \def\temp{#1}%
  \ifx\temp\macrocharA\Aogonek
  \else\ifx\temp\macrochara\aogonek
  \else\ifx\temp\macrocharE\Eogonek
  \else\ifx\temp\macrochare\eogonek
  \else
    \ecfont \setbox0=\hbox{#1}%
    \ifdim\ht0=1ex\accent"0C #1%
    \else\ooalign{\unhbox0\crcr\hidewidth\char"0C \hidewidth}%
    \fi
  \fi\fi\fi\fi
  }%
}
\def\Aogonek{{\ecfont \char"81}}\def\macrocharA{A}
\def\aogonek{{\ecfont \char"A1}}\def\macrochara{a}
\def\Eogonek{{\ecfont \char"86}}\def\macrocharE{E}
\def\eogonek{{\ecfont \char"A6}}\def\macrochare{e}
%
% Use the ec* fonts (cm-super in outline format) for non-CM glyphs.
\def\ecfont{%
  % We can't distinguish serif/sans and italic/slanted, but this
  % is used for crude hacks anyway (like adding French and German
  % quotes to documents typeset with CM, where we lose kerning), so
  % hopefully nobody will notice/care.
  \edef\ecsize{\csname\curfontsize ecsize\endcsname}%
  \edef\nominalsize{\csname\curfontsize nominalsize\endcsname}%
  \ifmonospace
    % typewriter:
    \font\thisecfont = ectt\ecsize \space at \nominalsize
  \else
    \ifx\curfontstyle\bfstylename
      % bold:
      \font\thisecfont = ecb\ifusingit{i}{x}\ecsize \space at \nominalsize
    \else
      % regular:
      \font\thisecfont = ec\ifusingit{ti}{rm}\ecsize \space at \nominalsize
    \fi
  \fi
  \thisecfont
}

% @registeredsymbol - R in a circle.  The font for the R should really
% be smaller yet, but lllsize is the best we can do for now.
% Adapted from the plain.tex definition of \copyright.
%
\def\registeredsymbol{%
  $^{{\ooalign{\hfil\raise.07ex\hbox{\selectfonts\lllsize R}%
               \hfil\crcr\Orb}}%
    }$%
}

% @textdegree - the normal degrees sign.
%
\def\textdegree{$^\circ$}

% Laurent Siebenmann reports \Orb undefined with:
%  Textures 1.7.7 (preloaded format=plain 93.10.14)  (68K)  16 APR 2004 02:38
% so we'll define it if necessary.
%
\ifx\Orb\thisisundefined
\def\Orb{\mathhexbox20D}
\fi

% Quotes.
\chardef\quotedblleft="5C
\chardef\quotedblright=`\"
\chardef\quoteleft=`\`
\chardef\quoteright=`\'


\message{page headings,}

\newskip\titlepagetopglue \titlepagetopglue = 1.5in
\newskip\titlepagebottomglue \titlepagebottomglue = 2pc

% First the title page.  Must do @settitle before @titlepage.
\newif\ifseenauthor
\newif\iffinishedtitlepage

% Do an implicit @contents or @shortcontents after @end titlepage if the
% user says @setcontentsaftertitlepage or @setshortcontentsaftertitlepage.
%
\newif\ifsetcontentsaftertitlepage
 \let\setcontentsaftertitlepage = \setcontentsaftertitlepagetrue
\newif\ifsetshortcontentsaftertitlepage
 \let\setshortcontentsaftertitlepage = \setshortcontentsaftertitlepagetrue

\parseargdef\shorttitlepage{%
  \begingroup \hbox{}\vskip 1.5in \chaprm \centerline{#1}%
  \endgroup\page\hbox{}\page}

\envdef\titlepage{%
  % Open one extra group, as we want to close it in the middle of \Etitlepage.
  \begingroup
    \parindent=0pt \textfonts
    % Leave some space at the very top of the page.
    \vglue\titlepagetopglue
    % No rule at page bottom unless we print one at the top with @title.
    \finishedtitlepagetrue
    %
    % Most title ``pages'' are actually two pages long, with space
    % at the top of the second.  We don't want the ragged left on the second.
    \let\oldpage = \page
    \def\page{%
      \iffinishedtitlepage\else
	 \finishtitlepage
      \fi
      \let\page = \oldpage
      \page
      \null
    }%
}

\def\Etitlepage{%
    \iffinishedtitlepage\else
	\finishtitlepage
    \fi
    % It is important to do the page break before ending the group,
    % because the headline and footline are only empty inside the group.
    % If we use the new definition of \page, we always get a blank page
    % after the title page, which we certainly don't want.
    \oldpage
  \endgroup
  %
  % Need this before the \...aftertitlepage checks so that if they are
  % in effect the toc pages will come out with page numbers.
  \HEADINGSon
  %
  % If they want short, they certainly want long too.
  \ifsetshortcontentsaftertitlepage
    \shortcontents
    \contents
    \global\let\shortcontents = \relax
    \global\let\contents = \relax
  \fi
  %
  \ifsetcontentsaftertitlepage
    \contents
    \global\let\contents = \relax
    \global\let\shortcontents = \relax
  \fi
}

\def\finishtitlepage{%
  \vskip4pt \hrule height 2pt width \hsize
  \vskip\titlepagebottomglue
  \finishedtitlepagetrue
}

% Settings used for typesetting titles: no hyphenation, no indentation,
% don't worry much about spacing, ragged right.  This should be used
% inside a \vbox, and fonts need to be set appropriately first.  Because
% it is always used for titles, nothing else, we call \rmisbold.  \par
% should be specified before the end of the \vbox, since a vbox is a group.
% 
\def\raggedtitlesettings{%
  \rmisbold
  \hyphenpenalty=10000
  \parindent=0pt
  \tolerance=5000
  \ptexraggedright
}

% Macros to be used within @titlepage:

\let\subtitlerm=\tenrm
\def\subtitlefont{\subtitlerm \normalbaselineskip = 13pt \normalbaselines}

\parseargdef\title{%
  \checkenv\titlepage
  \vbox{\titlefonts \raggedtitlesettings #1\par}%
  % print a rule at the page bottom also.
  \finishedtitlepagefalse
  \vskip4pt \hrule height 4pt width \hsize \vskip4pt
}

\parseargdef\subtitle{%
  \checkenv\titlepage
  {\subtitlefont \rightline{#1}}%
}

% @author should come last, but may come many times.
% It can also be used inside @quotation.
%
\parseargdef\author{%
  \def\temp{\quotation}%
  \ifx\thisenv\temp
    \def\quotationauthor{#1}% printed in \Equotation.
  \else
    \checkenv\titlepage
    \ifseenauthor\else \vskip 0pt plus 1filll \seenauthortrue \fi
    {\secfonts\rmisbold \leftline{#1}}%
  \fi
}


% Set up page headings and footings.

\let\thispage=\folio

\newtoks\evenheadline    % headline on even pages
\newtoks\oddheadline     % headline on odd pages
\newtoks\evenfootline    % footline on even pages
\newtoks\oddfootline     % footline on odd pages

% Now make TeX use those variables
\headline={{\textfonts\rm \ifodd\pageno \the\oddheadline
                            \else \the\evenheadline \fi}}
\footline={{\textfonts\rm \ifodd\pageno \the\oddfootline
                            \else \the\evenfootline \fi}\HEADINGShook}
\let\HEADINGShook=\relax

% Commands to set those variables.
% For example, this is what  @headings on  does
% @evenheading @thistitle|@thispage|@thischapter
% @oddheading @thischapter|@thispage|@thistitle
% @evenfooting @thisfile||
% @oddfooting ||@thisfile


\def\evenheading{\parsearg\evenheadingxxx}
\def\evenheadingxxx #1{\evenheadingyyy #1\|\|\|\|\finish}
\def\evenheadingyyy #1\|#2\|#3\|#4\finish{%
\global\evenheadline={\rlap{\centerline{#2}}\line{#1\hfil#3}}}

\def\oddheading{\parsearg\oddheadingxxx}
\def\oddheadingxxx #1{\oddheadingyyy #1\|\|\|\|\finish}
\def\oddheadingyyy #1\|#2\|#3\|#4\finish{%
\global\oddheadline={\rlap{\centerline{#2}}\line{#1\hfil#3}}}

\parseargdef\everyheading{\oddheadingxxx{#1}\evenheadingxxx{#1}}%

\def\evenfooting{\parsearg\evenfootingxxx}
\def\evenfootingxxx #1{\evenfootingyyy #1\|\|\|\|\finish}
\def\evenfootingyyy #1\|#2\|#3\|#4\finish{%
\global\evenfootline={\rlap{\centerline{#2}}\line{#1\hfil#3}}}

\def\oddfooting{\parsearg\oddfootingxxx}
\def\oddfootingxxx #1{\oddfootingyyy #1\|\|\|\|\finish}
\def\oddfootingyyy #1\|#2\|#3\|#4\finish{%
  \global\oddfootline = {\rlap{\centerline{#2}}\line{#1\hfil#3}}%
  %
  % Leave some space for the footline.  Hopefully ok to assume
  % @evenfooting will not be used by itself.
  \global\advance\pageheight by -12pt
  \global\advance\vsize by -12pt
}

\parseargdef\everyfooting{\oddfootingxxx{#1}\evenfootingxxx{#1}}

% @evenheadingmarks top     \thischapter <- chapter at the top of a page
% @evenheadingmarks bottom  \thischapter <- chapter at the bottom of a page
%
% The same set of arguments for:
%
% @oddheadingmarks
% @evenfootingmarks
% @oddfootingmarks
% @everyheadingmarks
% @everyfootingmarks

\def\evenheadingmarks{\headingmarks{even}{heading}}
\def\oddheadingmarks{\headingmarks{odd}{heading}}
\def\evenfootingmarks{\headingmarks{even}{footing}}
\def\oddfootingmarks{\headingmarks{odd}{footing}}
\def\everyheadingmarks#1 {\headingmarks{even}{heading}{#1}
                          \headingmarks{odd}{heading}{#1} }
\def\everyfootingmarks#1 {\headingmarks{even}{footing}{#1}
                          \headingmarks{odd}{footing}{#1} }
% #1 = even/odd, #2 = heading/footing, #3 = top/bottom.
\def\headingmarks#1#2#3 {%
  \expandafter\let\expandafter\temp \csname get#3headingmarks\endcsname
  \global\expandafter\let\csname get#1#2marks\endcsname \temp
}

\everyheadingmarks bottom
\everyfootingmarks bottom

% @headings double      turns headings on for double-sided printing.
% @headings single      turns headings on for single-sided printing.
% @headings off         turns them off.
% @headings on          same as @headings double, retained for compatibility.
% @headings after       turns on double-sided headings after this page.
% @headings doubleafter turns on double-sided headings after this page.
% @headings singleafter turns on single-sided headings after this page.
% By default, they are off at the start of a document,
% and turned `on' after @end titlepage.

\def\headings #1 {\csname HEADINGS#1\endcsname}

\def\headingsoff{% non-global headings elimination
  \evenheadline={\hfil}\evenfootline={\hfil}%
   \oddheadline={\hfil}\oddfootline={\hfil}%
}

\def\HEADINGSoff{{\globaldefs=1 \headingsoff}} % global setting
\HEADINGSoff  % it's the default

% When we turn headings on, set the page number to 1.
% For double-sided printing, put current file name in lower left corner,
% chapter name on inside top of right hand pages, document
% title on inside top of left hand pages, and page numbers on outside top
% edge of all pages.
\def\HEADINGSdouble{%
\global\pageno=1
\global\evenfootline={\hfil}
\global\oddfootline={\hfil}
\global\evenheadline={\line{\folio\hfil\thistitle}}
\global\oddheadline={\line{\thischapter\hfil\folio}}
\global\let\contentsalignmacro = \chapoddpage
}
\let\contentsalignmacro = \chappager

% For single-sided printing, chapter title goes across top left of page,
% page number on top right.
\def\HEADINGSsingle{%
\global\pageno=1
\global\evenfootline={\hfil}
\global\oddfootline={\hfil}
\global\evenheadline={\line{\thischapter\hfil\folio}}
\global\oddheadline={\line{\thischapter\hfil\folio}}
\global\let\contentsalignmacro = \chappager
}
\def\HEADINGSon{\HEADINGSdouble}

\def\HEADINGSafter{\let\HEADINGShook=\HEADINGSdoublex}
\let\HEADINGSdoubleafter=\HEADINGSafter
\def\HEADINGSdoublex{%
\global\evenfootline={\hfil}
\global\oddfootline={\hfil}
\global\evenheadline={\line{\folio\hfil\thistitle}}
\global\oddheadline={\line{\thischapter\hfil\folio}}
\global\let\contentsalignmacro = \chapoddpage
}

\def\HEADINGSsingleafter{\let\HEADINGShook=\HEADINGSsinglex}
\def\HEADINGSsinglex{%
\global\evenfootline={\hfil}
\global\oddfootline={\hfil}
\global\evenheadline={\line{\thischapter\hfil\folio}}
\global\oddheadline={\line{\thischapter\hfil\folio}}
\global\let\contentsalignmacro = \chappager
}

% Subroutines used in generating headings
% This produces Day Month Year style of output.
% Only define if not already defined, in case a txi-??.tex file has set
% up a different format (e.g., txi-cs.tex does this).
\ifx\today\thisisundefined
\def\today{%
  \number\day\space
  \ifcase\month
  \or\putwordMJan\or\putwordMFeb\or\putwordMMar\or\putwordMApr
  \or\putwordMMay\or\putwordMJun\or\putwordMJul\or\putwordMAug
  \or\putwordMSep\or\putwordMOct\or\putwordMNov\or\putwordMDec
  \fi
  \space\number\year}
\fi

% @settitle line...  specifies the title of the document, for headings.
% It generates no output of its own.
\def\thistitle{\putwordNoTitle}
\def\settitle{\parsearg{\gdef\thistitle}}


\message{tables,}
% Tables -- @table, @ftable, @vtable, @item(x).

% default indentation of table text
\newdimen\tableindent \tableindent=.8in
% default indentation of @itemize and @enumerate text
\newdimen\itemindent  \itemindent=.3in
% margin between end of table item and start of table text.
\newdimen\itemmargin  \itemmargin=.1in

% used internally for \itemindent minus \itemmargin
\newdimen\itemmax

% Note @table, @ftable, and @vtable define @item, @itemx, etc., with
% these defs.
% They also define \itemindex
% to index the item name in whatever manner is desired (perhaps none).

\newif\ifitemxneedsnegativevskip

\def\itemxpar{\par\ifitemxneedsnegativevskip\nobreak\vskip-\parskip\nobreak\fi}

\def\internalBitem{\smallbreak \parsearg\itemzzz}
\def\internalBitemx{\itemxpar \parsearg\itemzzz}

\def\itemzzz #1{\begingroup %
  \advance\hsize by -\rightskip
  \advance\hsize by -\tableindent
  \setbox0=\hbox{\itemindicate{#1}}%
  \itemindex{#1}%
  \nobreak % This prevents a break before @itemx.
  %
  % If the item text does not fit in the space we have, put it on a line
  % by itself, and do not allow a page break either before or after that
  % line.  We do not start a paragraph here because then if the next
  % command is, e.g., @kindex, the whatsit would get put into the
  % horizontal list on a line by itself, resulting in extra blank space.
  \ifdim \wd0>\itemmax
    %
    % Make this a paragraph so we get the \parskip glue and wrapping,
    % but leave it ragged-right.
    \begingroup
      \advance\leftskip by-\tableindent
      \advance\hsize by\tableindent
      \advance\rightskip by0pt plus1fil\relax
      \leavevmode\unhbox0\par
    \endgroup
    %
    % We're going to be starting a paragraph, but we don't want the
    % \parskip glue -- logically it's part of the @item we just started.
    \nobreak \vskip-\parskip
    %
    % Stop a page break at the \parskip glue coming up.  However, if
    % what follows is an environment such as @example, there will be no
    % \parskip glue; then the negative vskip we just inserted would
    % cause the example and the item to crash together.  So we use this
    % bizarre value of 10001 as a signal to \aboveenvbreak to insert
    % \parskip glue after all.  Section titles are handled this way also.
    %
    \penalty 10001
    \endgroup
    \itemxneedsnegativevskipfalse
  \else
    % The item text fits into the space.  Start a paragraph, so that the
    % following text (if any) will end up on the same line.
    \noindent
    % Do this with kerns and \unhbox so that if there is a footnote in
    % the item text, it can migrate to the main vertical list and
    % eventually be printed.
    \nobreak\kern-\tableindent
    \dimen0 = \itemmax  \advance\dimen0 by \itemmargin \advance\dimen0 by -\wd0
    \unhbox0
    \nobreak\kern\dimen0
    \endgroup
    \itemxneedsnegativevskiptrue
  \fi
}

\def\item{\errmessage{@item while not in a list environment}}
\def\itemx{\errmessage{@itemx while not in a list environment}}

% @table, @ftable, @vtable.
\envdef\table{%
  \let\itemindex\gobble
  \tablecheck{table}%
}
\envdef\ftable{%
  \def\itemindex ##1{\doind {fn}{\code{##1}}}%
  \tablecheck{ftable}%
}
\envdef\vtable{%
  \def\itemindex ##1{\doind {vr}{\code{##1}}}%
  \tablecheck{vtable}%
}
\def\tablecheck#1{%
  \ifnum \the\catcode`\^^M=\active
    \endgroup
    \errmessage{This command won't work in this context; perhaps the problem is
      that we are \inenvironment\thisenv}%
    \def\next{\doignore{#1}}%
  \else
    \let\next\tablex
  \fi
  \next
}
\def\tablex#1{%
  \def\itemindicate{#1}%
  \parsearg\tabley
}
\def\tabley#1{%
  {%
    \makevalueexpandable
    \edef\temp{\noexpand\tablez #1\space\space\space}%
    \expandafter
  }\temp \endtablez
}
\def\tablez #1 #2 #3 #4\endtablez{%
  \aboveenvbreak
  \ifnum 0#1>0 \advance \leftskip by #1\mil \fi
  \ifnum 0#2>0 \tableindent=#2\mil \fi
  \ifnum 0#3>0 \advance \rightskip by #3\mil \fi
  \itemmax=\tableindent
  \advance \itemmax by -\itemmargin
  \advance \leftskip by \tableindent
  \exdentamount=\tableindent
  \parindent = 0pt
  \parskip = \smallskipamount
  \ifdim \parskip=0pt \parskip=2pt \fi
  \let\item = \internalBitem
  \let\itemx = \internalBitemx
}
\def\Etable{\endgraf\afterenvbreak}
\let\Eftable\Etable
\let\Evtable\Etable
\let\Eitemize\Etable
\let\Eenumerate\Etable

% This is the counter used by @enumerate, which is really @itemize

\newcount \itemno

\envdef\itemize{\parsearg\doitemize}

\def\doitemize#1{%
  \aboveenvbreak
  \itemmax=\itemindent
  \advance\itemmax by -\itemmargin
  \advance\leftskip by \itemindent
  \exdentamount=\itemindent
  \parindent=0pt
  \parskip=\smallskipamount
  \ifdim\parskip=0pt \parskip=2pt \fi
  %
  % Try typesetting the item mark that if the document erroneously says
  % something like @itemize @samp (intending @table), there's an error
  % right away at the @itemize.  It's not the best error message in the
  % world, but it's better than leaving it to the @item.  This means if
  % the user wants an empty mark, they have to say @w{} not just @w.
  \def\itemcontents{#1}%
  \setbox0 = \hbox{\itemcontents}%
  %
  % @itemize with no arg is equivalent to @itemize @bullet.
  \ifx\itemcontents\empty\def\itemcontents{\bullet}\fi
  %
  \let\item=\itemizeitem
}

% Definition of @item while inside @itemize and @enumerate.
%
\def\itemizeitem{%
  \advance\itemno by 1  % for enumerations
  {\let\par=\endgraf \smallbreak}% reasonable place to break
  {%
   % If the document has an @itemize directly after a section title, a
   % \nobreak will be last on the list, and \sectionheading will have
   % done a \vskip-\parskip.  In that case, we don't want to zero
   % parskip, or the item text will crash with the heading.  On the
   % other hand, when there is normal text preceding the item (as there
   % usually is), we do want to zero parskip, or there would be too much
   % space.  In that case, we won't have a \nobreak before.  At least
   % that's the theory.
   \ifnum\lastpenalty<10000 \parskip=0in \fi
   \noindent
   \hbox to 0pt{\hss \itemcontents \kern\itemmargin}%
   %
   \vadjust{\penalty 1200}}% not good to break after first line of item.
  \flushcr
}

% \splitoff TOKENS\endmark defines \first to be the first token in
% TOKENS, and \rest to be the remainder.
%
\def\splitoff#1#2\endmark{\def\first{#1}\def\rest{#2}}%

% Allow an optional argument of an uppercase letter, lowercase letter,
% or number, to specify the first label in the enumerated list.  No
% argument is the same as `1'.
%
\envparseargdef\enumerate{\enumeratey #1  \endenumeratey}
\def\enumeratey #1 #2\endenumeratey{%
  % If we were given no argument, pretend we were given `1'.
  \def\thearg{#1}%
  \ifx\thearg\empty \def\thearg{1}\fi
  %
  % Detect if the argument is a single token.  If so, it might be a
  % letter.  Otherwise, the only valid thing it can be is a number.
  % (We will always have one token, because of the test we just made.
  % This is a good thing, since \splitoff doesn't work given nothing at
  % all -- the first parameter is undelimited.)
  \expandafter\splitoff\thearg\endmark
  \ifx\rest\empty
    % Only one token in the argument.  It could still be anything.
    % A ``lowercase letter'' is one whose \lccode is nonzero.
    % An ``uppercase letter'' is one whose \lccode is both nonzero, and
    %   not equal to itself.
    % Otherwise, we assume it's a number.
    %
    % We need the \relax at the end of the \ifnum lines to stop TeX from
    % continuing to look for a <number>.
    %
    \ifnum\lccode\expandafter`\thearg=0\relax
      \numericenumerate % a number (we hope)
    \else
      % It's a letter.
      \ifnum\lccode\expandafter`\thearg=\expandafter`\thearg\relax
        \lowercaseenumerate % lowercase letter
      \else
        \uppercaseenumerate % uppercase letter
      \fi
    \fi
  \else
    % Multiple tokens in the argument.  We hope it's a number.
    \numericenumerate
  \fi
}

% An @enumerate whose labels are integers.  The starting integer is
% given in \thearg.
%
\def\numericenumerate{%
  \itemno = \thearg
  \startenumeration{\the\itemno}%
}

% The starting (lowercase) letter is in \thearg.
\def\lowercaseenumerate{%
  \itemno = \expandafter`\thearg
  \startenumeration{%
    % Be sure we're not beyond the end of the alphabet.
    \ifnum\itemno=0
      \errmessage{No more lowercase letters in @enumerate; get a bigger
                  alphabet}%
    \fi
    \char\lccode\itemno
  }%
}

% The starting (uppercase) letter is in \thearg.
\def\uppercaseenumerate{%
  \itemno = \expandafter`\thearg
  \startenumeration{%
    % Be sure we're not beyond the end of the alphabet.
    \ifnum\itemno=0
      \errmessage{No more uppercase letters in @enumerate; get a bigger
                  alphabet}
    \fi
    \char\uccode\itemno
  }%
}

% Call \doitemize, adding a period to the first argument and supplying the
% common last two arguments.  Also subtract one from the initial value in
% \itemno, since @item increments \itemno.
%
\def\startenumeration#1{%
  \advance\itemno by -1
  \doitemize{#1.}\flushcr
}

% @alphaenumerate and @capsenumerate are abbreviations for giving an arg
% to @enumerate.
%
\def\alphaenumerate{\enumerate{a}}
\def\capsenumerate{\enumerate{A}}
\def\Ealphaenumerate{\Eenumerate}
\def\Ecapsenumerate{\Eenumerate}


% @multitable macros
% Amy Hendrickson, 8/18/94, 3/6/96
%
% @multitable ... @end multitable will make as many columns as desired.
% Contents of each column will wrap at width given in preamble.  Width
% can be specified either with sample text given in a template line,
% or in percent of \hsize, the current width of text on page.

% Table can continue over pages but will only break between lines.

% To make preamble:
%
% Either define widths of columns in terms of percent of \hsize:
%   @multitable @columnfractions .25 .3 .45
%   @item ...
%
%   Numbers following @columnfractions are the percent of the total
%   current hsize to be used for each column. You may use as many
%   columns as desired.


% Or use a template:
%   @multitable {Column 1 template} {Column 2 template} {Column 3 template}
%   @item ...
%   using the widest term desired in each column.

% Each new table line starts with @item, each subsequent new column
% starts with @tab. Empty columns may be produced by supplying @tab's
% with nothing between them for as many times as empty columns are needed,
% ie, @tab@tab@tab will produce two empty columns.

% @item, @tab do not need to be on their own lines, but it will not hurt
% if they are.

% Sample multitable:

%   @multitable {Column 1 template} {Column 2 template} {Column 3 template}
%   @item first col stuff @tab second col stuff @tab third col
%   @item
%   first col stuff
%   @tab
%   second col stuff
%   @tab
%   third col
%   @item first col stuff @tab second col stuff
%   @tab Many paragraphs of text may be used in any column.
%
%         They will wrap at the width determined by the template.
%   @item@tab@tab This will be in third column.
%   @end multitable

% Default dimensions may be reset by user.
% @multitableparskip is vertical space between paragraphs in table.
% @multitableparindent is paragraph indent in table.
% @multitablecolmargin is horizontal space to be left between columns.
% @multitablelinespace is space to leave between table items, baseline
%                                                            to baseline.
%   0pt means it depends on current normal line spacing.
%
\newskip\multitableparskip
\newskip\multitableparindent
\newdimen\multitablecolspace
\newskip\multitablelinespace
\multitableparskip=0pt
\multitableparindent=6pt
\multitablecolspace=12pt
\multitablelinespace=0pt

% Macros used to set up halign preamble:
%
\let\endsetuptable\relax
\def\xendsetuptable{\endsetuptable}
\let\columnfractions\relax
\def\xcolumnfractions{\columnfractions}
\newif\ifsetpercent

% #1 is the @columnfraction, usually a decimal number like .5, but might
% be just 1.  We just use it, whatever it is.
%
\def\pickupwholefraction#1 {%
  \global\advance\colcount by 1
  \expandafter\xdef\csname col\the\colcount\endcsname{#1\hsize}%
  \setuptable
}

\newcount\colcount
\def\setuptable#1{%
  \def\firstarg{#1}%
  \ifx\firstarg\xendsetuptable
    \let\go = \relax
  \else
    \ifx\firstarg\xcolumnfractions
      \global\setpercenttrue
    \else
      \ifsetpercent
         \let\go\pickupwholefraction
      \else
         \global\advance\colcount by 1
         \setbox0=\hbox{#1\unskip\space}% Add a normal word space as a
                   % separator; typically that is always in the input, anyway.
         \expandafter\xdef\csname col\the\colcount\endcsname{\the\wd0}%
      \fi
    \fi
    \ifx\go\pickupwholefraction
      % Put the argument back for the \pickupwholefraction call, so
      % we'll always have a period there to be parsed.
      \def\go{\pickupwholefraction#1}%
    \else
      \let\go = \setuptable
    \fi%
  \fi
  \go
}

% multitable-only commands.
%
% @headitem starts a heading row, which we typeset in bold.
% Assignments have to be global since we are inside the implicit group
% of an alignment entry.  \everycr resets \everytab so we don't have to
% undo it ourselves.
\def\headitemfont{\b}% for people to use in the template row; not changeable
\def\headitem{%
  \checkenv\multitable
  \crcr
  \global\everytab={\bf}% can't use \headitemfont since the parsing differs
  \the\everytab % for the first item
}%
%
% A \tab used to include \hskip1sp.  But then the space in a template
% line is not enough.  That is bad.  So let's go back to just `&' until
% we again encounter the problem the 1sp was intended to solve.
%					--karl, nathan@acm.org, 20apr99.
\def\tab{\checkenv\multitable &\the\everytab}%

% @multitable ... @end multitable definitions:
%
\newtoks\everytab  % insert after every tab.
%
\envdef\multitable{%
  \vskip\parskip
  \startsavinginserts
  %
  % @item within a multitable starts a normal row.
  % We use \def instead of \let so that if one of the multitable entries
  % contains an @itemize, we don't choke on the \item (seen as \crcr aka
  % \endtemplate) expanding \doitemize.
  \def\item{\crcr}%
  %
  \tolerance=9500
  \hbadness=9500
  \setmultitablespacing
  \parskip=\multitableparskip
  \parindent=\multitableparindent
  \overfullrule=0pt
  \global\colcount=0
  %
  \everycr = {%
    \noalign{%
      \global\everytab={}%
      \global\colcount=0 % Reset the column counter.
      % Check for saved footnotes, etc.
      \checkinserts
      % Keeps underfull box messages off when table breaks over pages.
      %\filbreak
	% Maybe so, but it also creates really weird page breaks when the
	% table breaks over pages. Wouldn't \vfil be better?  Wait until the
	% problem manifests itself, so it can be fixed for real --karl.
    }%
  }%
  %
  \parsearg\domultitable
}
\def\domultitable#1{%
  % To parse everything between @multitable and @item:
  \setuptable#1 \endsetuptable
  %
  % This preamble sets up a generic column definition, which will
  % be used as many times as user calls for columns.
  % \vtop will set a single line and will also let text wrap and
  % continue for many paragraphs if desired.
  \halign\bgroup &%
    \global\advance\colcount by 1
    \multistrut
    \vtop{%
      % Use the current \colcount to find the correct column width:
      \hsize=\expandafter\csname col\the\colcount\endcsname
      %
      % In order to keep entries from bumping into each other
      % we will add a \leftskip of \multitablecolspace to all columns after
      % the first one.
      %
      % If a template has been used, we will add \multitablecolspace
      % to the width of each template entry.
      %
      % If the user has set preamble in terms of percent of \hsize we will
      % use that dimension as the width of the column, and the \leftskip
      % will keep entries from bumping into each other.  Table will start at
      % left margin and final column will justify at right margin.
      %
      % Make sure we don't inherit \rightskip from the outer environment.
      \rightskip=0pt
      \ifnum\colcount=1
	% The first column will be indented with the surrounding text.
	\advance\hsize by\leftskip
      \else
	\ifsetpercent \else
	  % If user has not set preamble in terms of percent of \hsize
	  % we will advance \hsize by \multitablecolspace.
	  \advance\hsize by \multitablecolspace
	\fi
       % In either case we will make \leftskip=\multitablecolspace:
      \leftskip=\multitablecolspace
      \fi
      % Ignoring space at the beginning and end avoids an occasional spurious
      % blank line, when TeX decides to break the line at the space before the
      % box from the multistrut, so the strut ends up on a line by itself.
      % For example:
      % @multitable @columnfractions .11 .89
      % @item @code{#}
      % @tab Legal holiday which is valid in major parts of the whole country.
      % Is automatically provided with highlighting sequences respectively
      % marking characters.
      \noindent\ignorespaces##\unskip\multistrut
    }\cr
}
\def\Emultitable{%
  \crcr
  \egroup % end the \halign
  \global\setpercentfalse
}

\def\setmultitablespacing{%
  \def\multistrut{\strut}% just use the standard line spacing
  %
  % Compute \multitablelinespace (if not defined by user) for use in
  % \multitableparskip calculation.  We used define \multistrut based on
  % this, but (ironically) that caused the spacing to be off.
  % See bug-texinfo report from Werner Lemberg, 31 Oct 2004 12:52:20 +0100.
\ifdim\multitablelinespace=0pt
\setbox0=\vbox{X}\global\multitablelinespace=\the\baselineskip
\global\advance\multitablelinespace by-\ht0
\fi
% Test to see if parskip is larger than space between lines of
% table. If not, do nothing.
%        If so, set to same dimension as multitablelinespace.
\ifdim\multitableparskip>\multitablelinespace
\global\multitableparskip=\multitablelinespace
\global\advance\multitableparskip-7pt % to keep parskip somewhat smaller
                                      % than skip between lines in the table.
\fi%
\ifdim\multitableparskip=0pt
\global\multitableparskip=\multitablelinespace
\global\advance\multitableparskip-7pt % to keep parskip somewhat smaller
                                      % than skip between lines in the table.
\fi}


\message{conditionals,}

% @iftex, @ifnotdocbook, @ifnothtml, @ifnotinfo, @ifnotplaintext,
% @ifnotxml always succeed.  They currently do nothing; we don't
% attempt to check whether the conditionals are properly nested.  But we
% have to remember that they are conditionals, so that @end doesn't
% attempt to close an environment group.
%
\def\makecond#1{%
  \expandafter\let\csname #1\endcsname = \relax
  \expandafter\let\csname iscond.#1\endcsname = 1
}
\makecond{iftex}
\makecond{ifnotdocbook}
\makecond{ifnothtml}
\makecond{ifnotinfo}
\makecond{ifnotplaintext}
\makecond{ifnotxml}

% Ignore @ignore, @ifhtml, @ifinfo, and the like.
%
\def\direntry{\doignore{direntry}}
\def\documentdescription{\doignore{documentdescription}}
\def\docbook{\doignore{docbook}}
\def\html{\doignore{html}}
\def\ifdocbook{\doignore{ifdocbook}}
\def\ifhtml{\doignore{ifhtml}}
\def\ifinfo{\doignore{ifinfo}}
\def\ifnottex{\doignore{ifnottex}}
\def\ifplaintext{\doignore{ifplaintext}}
\def\ifxml{\doignore{ifxml}}
\def\ignore{\doignore{ignore}}
\def\menu{\doignore{menu}}
\def\xml{\doignore{xml}}

% Ignore text until a line `@end #1', keeping track of nested conditionals.
%
% A count to remember the depth of nesting.
\newcount\doignorecount

\def\doignore#1{\begingroup
  % Scan in ``verbatim'' mode:
  \obeylines
  \catcode`\@ = \other
  \catcode`\{ = \other
  \catcode`\} = \other
  %
  % Make sure that spaces turn into tokens that match what \doignoretext wants.
  \spaceisspace
  %
  % Count number of #1's that we've seen.
  \doignorecount = 0
  %
  % Swallow text until we reach the matching `@end #1'.
  \dodoignore{#1}%
}

{ \catcode`_=11 % We want to use \_STOP_ which cannot appear in